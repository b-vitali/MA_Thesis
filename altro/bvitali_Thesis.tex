\documentclass[12pt,a4paper,openright, oneside]{book}
\linespread{1.5}
\usepackage[greek,italian]{babel}
\usepackage[T1]{fontenc}
\usepackage[utf8]{inputenc}
\usepackage{amsmath}
\usepackage{amsfonts}
\usepackage{amssymb}
\title{Endimione, Capitolo 1}
\author{Anna Chiara Corradino}
\date{18/06/2017}
\usepackage[backend=biber, style=authoryear]{biblatex}
\addbibresource{Introduzione.bib}
\DeclareBibliographyCategory{Strum}
\DeclareBibliographyCategory{Fon}
\DeclareBibliographyCategory{FonB}
\defbibheading{Strum}{\subsection*{Strumenti}}
\defbibheading{Fon}{\subsection*{Fonti Antiche}}
\defbibheading{FonB}{\subsection*{Fonti Moderne}}
\renewcommand*{\newunitpunct}{\addcomma\space}
 \renewcommand*{\mkbibnamelast}[1]{\textsc{#1}} 
\DeclareFieldFormat[article]{title}{\textit{#1}}
\DeclareFieldFormat[book]{title}{\textit{#1}}
 \DeclareFieldFormat [article] {journaltitle}{{#1}}
  \DeclareFieldFormat[incollection]{title}{\mkbibemph{#1}}
  \DeclareFieldFormat[inbook]{title}{\textit{#1}}
\usepackage{makeidx}
\makeindex
\usepackage{geometry}
\geometry{a4paper, inner=4cm, outer=2.5cm, top=3cm, bmargin=2.5cm}
\usepackage{fancyhdr}
\pagestyle{fancy}
\usepackage{multicol}
\setlength{\columnsep}{1.5cm}
\usepackage{multirow}
\usepackage{booktabs}
\usepackage{tablefootnote}
\usepackage{tabularx}
\usepackage[errorshow]{longtable}
\usepackage{array}
\usepackage{setspace} %Non caricare setspace se vuoi mettere interline 1.5 alle note%
\usepackage{teubner}
\usepackage{acronym}
\begin{document}
\maketitle
\thispagestyle{empty}
\renewcommand{\contentsname}{\textbf{Sommario}}
{\linespread{1}{\tableofcontents}}

\addcontentsline{toc}{chapter}{Introduzione}

\chapter*{Introduzione}
\rhead{\textit{INTRODUZIONE}}
\renewcommand{\thechapter}{\Roman{chapter}}
\renewcommand{\thesection}{\thechapter \Roman{section}}
Studiare che o che cosa, decidere l'argomento e come affrontarlo sono i preliminari per la scrittura della tesi. Fondamentali infatti in un primo momento il porsi delle domande. \textit{In primis} perché ci interessi un argomento, e cosa ci spinge ad analizzarlo; in secondo luogo perché questo argomento sia importante al di fuori del personale interesse. Studiare che o che cosa, decidere l'argomento e come affrontarlo sono i preliminari per la scrittura della tesi. Fondamentali infatti in un primo momento il porsi delle domande. \textit{In primis} perché ci interessi un argomento, e cosa ci spinge ad analizzarlo; in secondo luogo perché questo argomento sia importante al di fuori del personale interesse. Studiare che o che cosa, decidere l'argomento e come affrontarlo sono i preliminari per la scrittura della tesi. Fondamentali infatti in un primo momento il porsi delle domande. \textit{In primis} perché ci interessi un argomento, e cosa ci spinge ad analizzarlo; in secondo luogo perché questo argomento sia importante al di fuori del personale interesse. 
Studiare che o che cosa, decidere l'argomento e come affrontarlo sono i preliminari per la scrittura della tesi. Fondamentali infatti in un primo momento il porsi delle domande. \textit{In primis} perché ci interessi un argomento, e cosa ci spinge ad analizzarlo; in secondo luogo perché questo argomento sia importante al di fuori del personale interesse. Studiare che o che cosa, decidere l'argomento e come affrontarlo sono i preliminari per la scrittura della tesi. Fondamentali infatti in un primo momento il porsi delle domande. \textit{In primis} perché ci interessi un argomento, e cosa ci spinge ad analizzarlo; in secondo luogo perché questo argomento sia importante al di fuori del personale interesse. Studiare che o che cosa, decidere l'argomento e come affrontarlo sono i preliminari per la scrittura della tesi. Fondamentali infatti in un primo momento il porsi delle domande. \textit{In primis} perché ci interessi un argomento, e cosa ci spinge ad analizzarlo; in secondo luogo perché questo argomento sia importante al di fuori del personale interesse. 

Studiare che o che cosa, decidere l'argomento e come affrontarlo sono i preliminari per la scrittura della tesi. Fondamentali infatti in un primo momento il porsi delle domande. \textit{In primis} perché ci interessi un argomento, e cosa ci spinge ad analizzarlo; in secondo luogo perché questo argomento sia importante al di fuori del personale interesse. Studiare che o che cosa, decidere l'argomento e come affrontarlo sono i preliminari per la scrittura della tesi. Fondamentali infatti in un primo momento il porsi delle domande. \textit{In primis} perché ci interessi un argomento, e cosa ci spinge ad analizzarlo; in secondo luogo perché questo argomento sia importante al di fuori del personale interesse. Studiare che o che cosa, decidere l'argomento e come affrontarlo sono i preliminari per la scrittura della tesi. Fondamentali infatti in un primo momento il porsi delle domande. \textit{In primis} perché ci interessi un argomento, e cosa ci spinge ad analizzarlo; in secondo luogo perché questo argomento sia importante al di fuori del personale interesse. 


\chapter*{Cavolfiori}
\rhead{\textit{INTRODUZIONE}}
\renewcommand{\thechapter}{\Roman{chapter}}
\renewcommand{\thesection}{\thechapter \Roman{section}}

Studiare che o che cosa, decidere l'argomento e come affrontarlo sono i preliminari per la scrittura della tesi. Fondamentali infatti in un primo momento il porsi delle domande. \textit{In primis} perché ci interessi un argomento, e cosa ci spinge ad analizzarlo; in secondo luogo perché questo argomento sia importante al di fuori del personale interesse. Studiare che o che cosa, decidere l'argomento e come affrontarlo sono i preliminari per la scrittura della tesi. Fondamentali infatti in un primo momento il porsi delle domande. \textit{In primis} perché ci interessi un argomento, e cosa ci spinge ad analizzarlo; in secondo luogo perché questo argomento sia importante al di fuori del personale interesse. 
Studiare che o che cosa, decidere l'argomento e come affrontarlo sono i preliminari per la scrittura della tesi. Fondamentali infatti in un primo momento il porsi delle domande. \textit{In primis} perché ci interessi un argomento, e cosa ci spinge ad analizzarlo; in secondo luogo perché questo argomento sia importante al di fuori del personale interesse. 

Studiare che o che cosa, decidere l'argomento e come affrontarlo sono i preliminari per la scrittura della tesi. Fondamentali infatti in un primo momento il porsi delle domande. \textit{In primis} perché ci interessi un argomento, e cosa ci spinge ad analizzarlo; in secondo luogo perché questo argomento sia importante al di fuori del personale interesse. 

Studiare che o che cosa, decidere l'argomento e come affrontarlo sono i preliminari per la scrittura della tesi. Fondamentali infatti in un primo momento il porsi delle domande. \textit{In primis} perché ci interessi un argomento, e cosa ci spinge ad analizzarlo; in secondo luogo perché questo argomento sia importante al di fuori del personale interesse. Studiare che o che cosa, decidere l'argomento e come affrontarlo sono i preliminari per la scrittura della tesi. Fondamentali infatti in un primo momento il porsi delle domande. \textit{In primis} perché ci interessi un argomento, e cosa ci spinge ad analizzarlo; in secondo luogo perché questo argomento sia importante al di fuori del personale interesse. Studiare che o che cosa, decidere l'argomento e come affrontarlo sono i preliminari per la scrittura della tesi. Fondamentali infatti in un primo momento il porsi delle domande. \textit{In primis} perché ci interessi un argomento, e cosa ci spinge ad analizzarlo; in secondo luogo perché questo argomento sia importante al di fuori del personale interesse. Studiare che o che cosa, decidere l'argomento e come affrontarlo sono i preliminari per la scrittura della tesi. Fondamentali infatti in un primo momento il porsi delle domande. \textit{In primis} perché ci interessi un argomento, e cosa ci spinge ad analizzarlo; in secondo luogo perché questo argomento sia importante al di fuori del personale interesse. 
\end{document}