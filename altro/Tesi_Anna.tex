
\documentclass[12pt,a4paper,openright, oneside]{book}
\linespread{1.5}
\usepackage[greek,italian]{babel}
\usepackage[T1]{fontenc}
\usepackage[utf8]{inputenc}
\usepackage{amsmath}
\usepackage{amsfonts}
\usepackage{amssymb}
\title{Endimione, Capitolo 1}
\author{Anna Chiara Corradino}
\date{18/06/2017}
\usepackage[backend=biber, style=authoryear]{biblatex}
\addbibresource{Introduzione.bib}
\DeclareBibliographyCategory{Strum}
\DeclareBibliographyCategory{Fon}
\DeclareBibliographyCategory{FonB}
\defbibheading{Strum}{\subsection*{Strumenti}}
\defbibheading{Fon}{\subsection*{Fonti Antiche}}
\defbibheading{FonB}{\subsection*{Fonti Moderne}}
\renewcommand*{\newunitpunct}{\addcomma\space}
 \renewcommand*{\mkbibnamelast}[1]{\textsc{#1}} 
\DeclareFieldFormat[article]{title}{\textit{#1}}
\DeclareFieldFormat[book]{title}{\textit{#1}}
 \DeclareFieldFormat [article] {journaltitle}{{#1}}
  \DeclareFieldFormat[incollection]{title}{\mkbibemph{#1}}
  \DeclareFieldFormat[inbook]{title}{\textit{#1}}
\usepackage{makeidx}
\makeindex
\usepackage{geometry}
\geometry{a4paper, inner=4cm, outer=2.5cm, top=3cm, bmargin=2.5cm}
\usepackage{fancyhdr}
\pagestyle{fancy}
\usepackage{multicol}
\setlength{\columnsep}{1.5cm}
\usepackage{multirow}
\usepackage{booktabs}
\usepackage{tablefootnote}
\usepackage{tabularx}
\usepackage[errorshow]{longtable}
\usepackage{array}
\usepackage{setspace} %Non caricare setspace se vuoi mettere interline 1.5 alle note%
\usepackage{teubner}
\usepackage{acronym}
\begin{document}
\maketitle
\thispagestyle{empty}
\renewcommand{\contentsname}{\textbf{Sommario}}
{\linespread{1}{\tableofcontents}}

\addcontentsline{toc}{chapter}{Introduzione}

\chapter*{Introduzione}
\rhead{\textit{INTRODUZIONE}}
\renewcommand{\thechapter}{\Roman{chapter}}
\renewcommand{\thesection}{\thechapter \Roman{section}}
Studiare che o che cosa, decidere l'argomento e come affrontarlo sono i preliminari per la scrittura della tesi. Fondamentali infatti in un primo momento il porsi delle domande. \textit{In primis} perché ci interessi un argomento, e cosa ci spinge ad analizzarlo; in secondo luogo perché questo argomento sia importante al di fuori del personale interesse. 

Perché dunque una ricerca sul mito di Endimione? Ma soprattutto di un Endimione così moderno?  La mia domanda iniziale in realtà riguardava la genesi dell'esplosione del mito di Endimione nel Rinascimento. Nel Rinascimento infatti si assiste a un'evidente ripresa e diffusione di miti minori del panorama classico. Il mondo rinascimentale conosce infatti spesse volte approfondimenti su tematiche e storie considerate `secondarie' per un innato gusto di erudizione foraggiato da due motivi principali: quella che, spesso ed erroneamente, viene chiamata la ``riscoperta dei classici'', che si tratta invece di una ``circolazione di classici'' grazie alla diffusione di supporti scrittorii più agili (la stampa, le Università con i loro sistemi di copiatura professionalizzati) e a una accelerazione del sistema postale europeo (che permise la possibilità di concepire una \textit{res publica litterarum} europea); una equivalenza fondamentale tra cultura (non solo pagana antica, ma molto più generale) e prestigio, al di fuori del chiostro.

Endimione si inserisce a pieno diritto nel panorama rinascimentale; è infatti un mito minore, dalla storia flessibile e a tratti romantica. Sarà infatti utilizzato come paradigma nelle più disparate situazioni, dall'Endimione astrologo e scopritore delle stelle, a un Endimione più romantico di cui si innamora la Luna e che diventa fertile paradigma poetico e filosofico, ancora a un Endimione come immagine di falsa coscienza. 

Le ``storie'' di Endimione infatti si presentano, come vedremo, come più letture e filoni che non saranno analizzati tutti per la necessità di circoscrivere il campo e per porre le basi iniziali di una ricerca potenzialmente più vasta.

Una prima circoscrizione è temporale. Si è deciso di escludere a priori le origini più antiche del mito, per evitare di perdersi nei meandri della tradizione, ma soprattutto per non far diventare il lavoro una ricerca di storia delle religioni o di antropologia del mondo antico, cosa che richiederebbe un approfondimento a sé, per vastità del materiale non solo letterario, ma anche archeologico e iconografico, e abilità differenti. Si è dunque deciso di restringere il campo alla tradizione rinascimentale cinquecentesca, per cui sarà interessante comprendere quale siano le fonti principali che abbiano influenzato e raggiunto il periodo storico di nostro interesse. La ricerca dunque si presenta come sincronica, nel rinascimento, e allo stesso diacronica per la breve storia della tradizione.

Una seconda restrizione è in merito alla materia da trattare. Il precipuo interesse del lavoro era di analizzare l'intera fortuna letteraria del mito di Endimione nel mondo moderno. Grazie al professor Alessandro Grilli, che mi ha seguito nel percorso, mi sono resa conto di quanto fosse ingenua una prospettiva così onnicomprensiva e vasta e di quanto sarebbe stato più interessante l'analisi di un solo aspetto del mito in un dato periodo e in un dato posto. Una ricerca dunque che, come spiegherò nel primo capitolo, ricade nel campo della tematologia, anche nel senso etimologico del termine di <<ciò che è dato>>.

Una terza e ultima restrizione, sebbene tutte e tre strettamente legate tra di loro, riguarda la scelta dei testi di analizzare. Si è deciso di infatti escludere, se non quando necessario, i testi musicali e le produzioni artistiche, e di concentrarsi solo sulla produzione filosofica (la trattatistica in particolare) e letteraria. 

\section{La divisione in capitoli}

I capitoli del presente elaborato sono tre.
Nel primo capitolo sono analizzate brevemente le linee metodologiche principali riguardanti lo studio sui miti e sulla loro ricezione. Infatti si sta lavorando in un campo molto vasto e ancora, forse fortunatamente, scevro da una metodologia univoca e precisa. Si è sentita dunque la necessità di indagare i filoni teorici più importanti della letteratura comparta sulla mitologia per poter comprendere quale fosse la o le linee teoriche adattabili al lavoro che ci si era prefisso di attuare.

Si è deciso, per soluzione di continuità, di unire due capitoli che inizialmente erano stati concepiti separatamente. Il capitolo 2 infatti si concentra sulla tradizione `endimionica' antica, tracciandone le linee fondamentali di diffusione rilevanti per l'analisi rinascimentale del mito, e sulle prime attestazioni rinascimentali. Mi riferirò infatti ad alcune tradizioni enciclopediche che tramandano il mito di Endimione fino all'entrata nel XVI secolo. In questo capitolo si è cercato di rispondere alla domanda fondamentale sulla ricezione specifica del mito in questione.

È chiaro infatti che non possiamo talvolta e non dobbiamo chiedere ai testi che leggiamo quali siano tutte le effettive fonti di un singolo autore o, come nel nostro caso, di una intera epoca storica. Un lavoro del genere oltre che borioso nei confronti dei testi, può risultare frustrante e meramente compilativo. Questo infatti distingue una lavoro ragionato sulle fonti importanti dalle opere, fondamentali, enciclopediche. È indubbio che la ricerca della citazione classica soprattutto per testi relativamente poco studiati, come quelli rinascimentali, sia molto attraente per lo studioso, tuttavia per un lavoro letterario non è sempre totalmente pertinente; in parte per la perdita effettiva di alcune fonti antiche, e di conseguenza l'impossibilità di ricostruire un `originale' della fonte, per usare un concetto caro alla filologia. In parte perché quando si tratta di una `trama' di un mito, le fonti stesse possono essere molteplici e cadere sotto il nome di più autori.

Di fronte a queste problematiche si è deciso di riportare i testi che sono sembrati più calzanti per la ricostruzione di una o più fonti-subarchetipo, con la consapevolezza del limite intrinseco del subarchetipo, che sempre presenterà errori. Questo perché ciò che davvero interessa in questa sede non è tanto l'originale, già scartato a priori, bensì la storia della tradizione che porta ai prodotti analizzati. 

Nel terzo capitolo sono analizzati i testi più rilevanti del '500 con specifico \textit{focus} per i testi filosofici e letterari che presentano Endimione, amante e amato di Diana, come paradigma `filosofico'. Mi riferisco in particolare a una delle principali riletture del mito nel '500. Sono partita infatti da un approdo mediano, quello di Giulio Camillo Delminio nell'\textit{Idea del Theatro}, perché ho avuto il sentore che l'autore, spesse volte bistrattato, abbia colto il senso profondo di uno dei filoni del racconto del mito. Endimione si presenta infatti come il risultato di una tradizione vasta, ma allo stesso tempo esso è il culmine dell'elaborazione di un tema caro anche al medioevo, della ``morte attraverso il bacio''.

Nelle conclusioni sono tirate le fila dell'intero lavoro, che si presenta chiaramente come aperto e da portare avanti in futuro. Mi premeva soprattutto stabilire le linee metodologiche da seguire e il principale filone di interesse. Di certo un lavoro completo sul mito di Endimione non è stato ancora portato a termine, sebbene alcuni tentativi siano stati fatti. Proprio per la sua peculiarità tutta, o quasi, rinascimentale, potrebbe essere foriero di conclusioni interessanti che qui sono presentate solo in maniera parziale, ma che attendono un approfondimento e una analisi onnicomprensiva della produzione non solo letteraria, ma anche iconografica e musicale.

\renewcommand{\thechapter}{\arabic{chapter}} 
\chapter{Un problema di metodologia}
\renewcommand{\thesection}{\thechapter. \arabic{section}}
\rhead{\textit{1. METODOLOGIA}}
Per esortare gli studiosi di comparatistica ad avere cautela nello studio  dei miti, sosteneva Brunel, che fosse necessario scartare in primo luogo il rischio di confusione di termini, orientamenti di ricerca, punti di partenza, di arrivo e di prospettiva. Il comparatista deve infatti definire \textit{in primis} l'oggetto della propria ricerca:

\begin{quote}\begin{singlespace}
\footnotesize{Préalablement à toute enqu\^ete t\^atonnante dans le poussière des bibliothèques, le comparatiste doit définir son objet, les termes qu'il emploie et se définir lui-m\^eme. Faute de quoi il restera un mythe, --au mauvais sens du terme.}\footnote{\cite{Brunel1}, p. 37.}\end{singlespace}\end{quote}
Proprio per evitare questa confusione è necessario stabilire quale sia il metodo che si è deciso di adottare, tenendo conto dei filoni metodologici esistenti che possano fornire validi spunti di ricerca. Il titolo del capitolo è, già di per sé, la spia di una metodologia che talvolta si presenta come problematica per mancanza di organicità.  Studiare un mito, anzi la sua ricezione, è un lavoro molto vasto che richiede forti restrizioni di campo, soprattutto in casi come quelli di Endimione per cui la quantità di materiale analizzabile può sembrare ingestibile, e soprattutto molto eterogenea. Far ciò senza linee guida è ancora più complicato, ed è il motivo per cui fin dall'introduzione si è cercato di circoscrivere il campo il più possibile. 

Il passo fondamentale della scelta della metodologia da seguire si presenta anch'esso come una parziale restrizione di campo. Scegliere infatti il metodo significa selezionare  il materiale pertinente. A questo si aggiunge il fatto che il metodo deve essere tagliato \textit{ad hoc} per il tema che si vuole problematizzare. Anche nelle discipline più formalizzate si sente sempre una necessità di riflessione caso per caso.

 Si ritiene inoltre necessario fornire alcune indicazioni che sono state i mattoni costitutivi delle fondamenta della nostra ricerca, perché è necessario rendere chiari alcuni concetti chiave, che, pur sembrando di banale comprensione, sono in realtà rilsultati ben ostici da affrontare. Molte domande che mi sono infatti posta prima di iniziare e che mi sembravano di facile risoluzione, sono risultate invece, non solo fondamentali, ma anche di non univoca definizione. Domande come per esempio cosa sia un mito, nozione di primo acchito banalmente comprensibile, ma non avente una effettiva risposta unica.  

La <<mitocritica>> (vd. § 1.4) è un campo relativamente nuovo, attivo e soprattutto vasto. Le più recenti pubblicazioni sul tema, ultima fra tutte l'edizione italiana della \textit{Mythocritque} di Pierre Brunel, curata Cettina Rizzo e Riccardo Raimondo nel 2015, testimoniano una viva attività di riflessione e ricerca.\footcite{Brunel} Il libro di Pierre Brunel (1939-), a cui noi dobbiamo una parte delle scelte metodologiche uscì nel 1992 e continua a fare scuola in fatto di mitocritica.\footcite{Brunel1}

La mitocritica moderna si presenta come un campo estremamente eterogeneo. Esso può essere semplicisticamente definito come un insieme di ricerche sulla mitologia. Tuttavia ciò che si intende oggi per mitocritica è un grande coacervo di studi che presenta molti sottinsiemi.
È dunque premura iniziale la selezione dei sottinsiemi pertinenti alla ricerca; in particolare si è deciso di selezionare due filoni principali: uno tematologico e uno <<archetipico>>.
 Riprendo in questa sede una distinzione fondamentale operata da Alessandro Grilli per la tradizione adonia, applicabile tuttavia a qualunque studio che voglia occuparsi di una tradizione mitica.

Un mito infatti è tramandato sotto più forme, e si possono individuare tre filoni di ricezione principali: una tradizione <<diegematica>>, una tradizione <<emblematica>> e una tradizione <<archetipica>>.\footnote{\cite{Grilli}, p. 24.} Questa distinzione preliminare è fondamentale per operare una scelta metodologica oculata.
La tradizione ``diegematica'' è la riscrittura del mito come modello letterario.
 La tradizione ``emblematica'',\footnote{Termine coniato da Alessandro Grilli sulla falsariga degli \textit{Emblemata} di Andrea Alciato, ivi, p. 23.} il cui studio si avvale degli strumenti della tematologia, e analizza il percorso di un mito letterario (o tema) come frutto di una consapevole scelta autoriale (l'autore cita esplicitamente il mito) per richiamare situazioni diverse dall'ambiente mitico di nascita. 
 Lo studio della tradizione ``archetipica'', infine, indaga la trasmissione dell'archetipo dell'immaginario soggiacente al mito studiato, che si propaga senza che ve ne sia una manifesta intenzione. Tale tradizione si avvale degli strumenti di ricerca che pertengono all'antropologia e alla psicanalisi.
 
In questo primo capitolo si è deciso di analizzare brevemente le metodologie correlate alle due tradizioni emblematica e archetipica, in quanto interesse specifico del presente lavoro. La tradizione diegematica invece è stata volutamente lasciata da parte perché il presente lavoro esaminerà solo sommariamente la tradizione del mito di Endimione come narrazione non estrapolata dal suo contesto. Ce ne serviremo infatti nella prima parte del secondo capitolo, quando sarà necessario comprendere come il mito di Diana ed Endimione sia entrato a far parte della sensibilità rinascimentale.

\section{Mito, una definizione semplice?}

È necessario definire una nozione fondamentale; finora si è parlato con leggerezza di un ``mito di Endimione'', tuttavia il termine ``mito'' non gode di una pacifica fortuna. Il significato è infatti intrinsecamente ambiguo e ancora più ambigue sono state le definizioni datene nel tempo, per l'uso e abuso del termine \textit{in saecula saeculorum}. La problematica infatti terminologica è stata, ed è tuttora, viva. Brunel, nella \textit{Mythocritique}, nel 1992, manifestava la mancanza di definizioni precise per quanto riguarda lo studio dei miti e partiva proprio dalla ricerca della definizione della parola ``mito'' per poter definire il campo di ricerca.\footnote{\cite{Brunel1}, pp. 13 e ss.}

Ritengo possa essere d'ausilio, per comprendere l'instabilità di una definizione `ballerina' come quella di ``mito'', consultare in primo luogo il dizionario della lingua italiana. Nel Battaglia troviamo scritto:
\begin{quote}\begin{singlespace}
\footnotesize{ \textbf{Mito}, sm. Narrazione simbolica di carattere sacrale che, in tempi e culture diverse, ha come personaggi divinità, eroi, antenati, mostri, animali: insieme con il rito costituisce un momento fondamentale dell'esperienza religiosa e tende a soddisfare il bisogno di fornire una spiegazione a fenomeni naturali o a problemi religiosi e di dare una legittimazione a pratiche cultuali o strutture sociali.--Anche l'evento simbolico che costituisce l'oggetto della narrazione o che ha per protagonista l'eroe, la divinità, il mostro o l'animale.\footnote{GDLI, ``Mito''.}}
\end{singlespace}\end{quote}

La definizione principale data da Battaglia, e affiancata da altre sette estensioni semantiche --dal mito collettivo alla funzione metaforica di mito in quanto oggetto non più esistente, ma considerato con ammirazione--, si focalizza su alcuni punti principali che devono essere chiariti. La definizione infatti è molto cauta e non si sbilancia in delimitazioni forti (d'altro canto questo sfumato è insito nei contorni della parola stessa). Il mito è \textit{in primis} una <<narrazione>>. Battaglia coglie nella definizione la sfumatura etimologica di ``\begin{otherlanguage}{greek} m\~ujoc\end{otherlanguage}'' che indicava inizialmente un mero ``racconto''  contrapposto al ``\begin{otherlanguage}{greek}l'ogoc \end{otherlanguage}''. Pur essendo effettivamente un racconto, non tutti i racconti tuttavia sono considerabili miti.\footnote{Contrariamente a quanto affermato da per esempio \cite{Wellek}.}

Vernant, postulando un'iniziale identificazione tra ``\textit{mythos}'' e ``\textit{logos}'', sosteneva che i due termini non fossero contrapposti, e portava a prova di ciò i composti ``\begin{otherlanguage}{greek}mujologe~in - mujolog'ia\end{otherlanguage}''.\footnote{Treccani-900, ``Mito'', a c. di Jean Pierre Vernant.} Tuttavia questi stessi composti manifestano una distanza semantica percepita come ontologica tra i due termini. Il mito contiene in sé qualcosa di irrazionale, sovrannaturale, che è stato sempre percepito come tale anche prima del ragionamento sul mito stesso.  Sottolinea infatti a ragione Battaglia che il mito è una narrazione <<simbolica>> e <<sacrale>>, perché da una parte è una narrazione che richiede uno sforzo intrepretativo, dall'altro il mito ha un contenuto `sovrannaurale' che lo caraterizza per il suo sfondo di irrazionalità.

Si aggiunga a questo che il mito è una narrazione che ha una caratteristica essenziale, identificata da Greimas come una <<ridondanza>>,\footnote{\cite{Greimas}, pp. 30-31.} ossia la capacità di reiterare delle forme fisse contemporaneamente alla prolificità generativa.
Sebbene Battaglia colga il significato pragmatico della parola ``mito'', c'è bisogno di un maggiore approfondimento su come venga e sia stato considerato il mito nella letteratura comparata. Non riprendo la definizione spesse volte citata di Mircea Eliade del termine, da una parte perché circondata da un alone romantico che ne offusca il reale senso, dall'altra perché come ha sostenuto anche Brunel, Mircea Eliade vede il mito come <<devalorizzazione del modello>>,\footnote{\cite{Brunel3}, p. 120.} che non ritengo sia accettabile, soprattutto per l'impostazione data alla presente ricerca. Mi sembra infatti più calzante per il nostro caso la definizione che troviamo nel manuale di letteratura comparata a cura di Didier Souiller e di Wladimir Troubetzkoy:
\begin{quote}\begin{singlespace}
\footnotesize{Le mythe est d'abord un ensemble narratif, composé d'éléments invariants et consacré par une tradition. A l'origine, ce récit, investi d'un pouvoir de fondation de valeurs d'un groupe ou d'une société, se caractérisait par une irruption du sacré ou du surnaturel; il servait ansi à marquer les bornes de l'humain et de la trascendence.}\footnote{\cite{Souiller}, p. 8.}
\end{singlespace}\end{quote}

I due studiosi sottolineano gli elementi interessanti del mito; è una narrazione sacra e simbolica, che presenta degli elementi <<invarianti>> perpetuati dalla tradizione. Aggiungono inoltre che questi miti non hanno senso di esistere senza la collettività che li accetta.

Ciò che tuttavia rimane fuori da questa definizione è una distinzione che è una delle pietre scottanti della critica novecentesca, cioè se sia effettivamente possibile parlare di un mito ``scritto'', ossia se si possa in qualche maniera conciliare la divergenza tra mito e discorso, o se invece il mito, quando `si fa letteratura', perda la sua sacralità e sia necessario parlare di un prodotto a sé. Armando Gnisci propone nel suo manuale di letteratura comparata,\footnote{\cite{Gnisci}, pp. 72-73.} di vedere la distinzione tra mito etno-religioso e mito letterario. La divisione è ripresa dal testo di Philippe Sellier all'interno di un breve articolo pubblicato nel 1984,\footcite{Sellier} nel quale lo studioso distingueva il mito etno-religioso e il mito letterario per poter, nelle conclusioni, evidenziarne i tratti comuni. 

Il mito etno-religioso è, secondo Sellier, quello che Ricoeur ha definito come <<instaurateur>>, anonimo, collettivo ed elaborato oralmente.\footnote{Ivi, p. 113.} Questo mito ha la funzione di collante sociale (è uno strumento a tutti gli effetti socio-religioso) nel quale i personaggi agiscono secondo la logica dell'immaginario, e sopratttutto è tramandato oralmente e coltivato mediante il rito. 

Il mito letterario\footnote{Definizione non di Sellier, ma di Pierre Albouy, come specificato nei prossimi paragrafi (\cite{Albouy}, p. 9).} è invece la trasformazione del mito in letteratura. Senza dare un giudizio negativo o positivo com'è stato fin troppo spesso fatto, il mito, nel momento in cui diventa scritto, cambia la sua fisionomia. E questo cambiamento è di vario genere e si riconnette a quelle tradizioni cha abbiamo citato sopra con le correlate metodologie di studio.

Si affianca a queste due definizioni il mito come archetipo dell'immaginario, ossia come costruzione mentale comune a tutti gli esseri viventi che si ripropone nelle sue forme più elementari indipendentemente dalla sua tradizione scritta o orale. Senza dunque un \textit{medium} consapevole, ma ancora più inconscio del mito etno-religioso.

Esistono dunque molteplici definizioni del termine ``mito'' sebbene come accezione di base sia accettabile la formulazione di Soullier e Troubetzkoy, in virtù della generalizzazione. Tuttavia, alla stregua di una copula di un predicato nominale, il mito necessita di un attributo per essere definito univocamente. Si ritiene dunque necessario passare alla problematica più strettamente metodologica. Dato per assodato infatti che si possa parlare di un ``mito di Endimione'' o di ``Diana ed Endimione''-- come ci sembra più corretto citare-- nella sua più vaga accezione di copula, e sarà  più interessante comprendere quali siano gli strumenti e i filoni metodologici pertinenti per analizzarne gli attributi e la tradizione.



\section{Andrè Jolles, \textit{Einfache Formen}}


Seguo ancora Brunel nella struttura di questo primo capitolo e vorrei dunque analizzare uno dei centri irradianti dello studio sulla mitologia che ha portato alla nascita della mitocritica stessa.

\textit{Einfache Formen}\footcite{Jolles} uscito nel 1930 e ristampato  più di sei volte in Germania (fino all'ultima edizione curata da De Gruyter nel 2006) propone di analizzare alcune forme letterarie tipiche della cultura popolare che, sia per semplicità strutturale, sia per brevità, appaiono come forme che si producono nel linguaggio senza l'intervento di un poeta (cfr. \cite{Jolles} ) La teoria di André Jolles (1874-1946), sebbene in alcuni punti criticata da Brunel, appare tuttora la base per chi si cimenta in lavori tematologici, morfologici e antropologici sui testi letterari. È dunque importante comprenderne le linee direttive.

La ricerca, morfologica, è volta a individuare delle forme preletterarie e autonome che confluiscono all'interno della letteratura, ma che sono \textit{a priori}, e che si oppongono diametralmente alla <<forme artistiche>>.

Nota bene Filippo Fonio\footnote{Ritengo opportuno sottolineare che insieme a Silvia Contarini, la curatrice italiana di \textit{Einfache Formen} di Jolles, proprio nel 2015 (1-2 luglio) hanno organizzato un convegno a Pisa interamente dedicato allo studioso olandese, i cui atti sono stati pubblicati online sul sito della rivista <<Cahiers d'études italiennes>> nel 2016, n. 23 (\cite{Fonio}).} quando sottolinea che però l'idea delle forme semplici non è originale jollesiano, ma frutto di una riflessione già romantica e neoclassica sulla natura estetica dei diversi stati dell'aggregazione artistica.\footnote{\cite{Fonio1}, pp. 152-3.} Si vedano per esempio le riflessioni di Vico o di Herder, sebbene molto lontane dalla formalizzazione jollesiana.

Jolles identifica nove <<forme semplici>>: Leggenda, Saga, Mito, Enigma, Massima, Caso Memorabile, Fiaba e Scherzo, considerate forme pure nel loro carattere fisso e atemporale. Sono quelle forme che non possono essere comprese né dalla retorica, né dalla Poetica, né dalla scrittura e che sono allo stesso tempo poesia ma non opera poetica.\footnote{\cite{Jolles}, p. 10.} Per comprendere cosa sia la forma semplice è tuttavia necessario partire dalle tre funzioni fondamentali del linguaggio di nominazione, fabbricazione e interpretazione che rispecchiano parallelamente tre funzioni fondamentali all'interno di una società: il contadino, l'artigiano e il sacerdote;\footnote{\cite{Jolles}, Einf\"uhrung, \textit{passim}.} il contadino ordina e inserisce la natura all'interno della vita dell'uomo (un lavoro di assoggettamento del selvaggio, dunque \textit{produce}). L'artigiano riorganizza ciò che in natura c'è privandolo della sua naturalità e servendosi di quanto è già stato prodotto (e dunque \textit{crea)}. Il sacerdote ha il compito di chiarire il senso dell'operato degli altri due, completandolo e integrandolo nella società.  Allo stesso modo il linguaggio, mediante la nominazione, produce (le parole), crea-compone (l'atto poetico) e interpreta-chiarifica, unificando la mutevolezza dei fenomeni sotto un unico segno.

Ciò che tuttavia a noi interessa nello specifico è chiaramente la trattazione del mito. Cos'è per Jolles il mito? Lamentava Brunel come Jolles tendesse a ridurre il mito solo al mito eziologico, scrive infatti Jolles, tentando di dare una definizione del termine mito, riprendendo e contestando la definizione di Jacob Grimm:
\begin{quote}\begin{singlespace}
\footnotesize Der Mensch fordert von der Welt und ihren Erscheinungen, da{\ss} sie sich ihm bekannt gebebn sollen. Und er bekommt \textit{Antwort}, das hei{\ss}t, er bekommt ihr Widerwort, ihr Wort tritt ihm entgegen. Die Welt und ihre Erscheinungen geben sich ihm bekannt. Wo sich nun in dieser Weise aus \textit{Frage} und \textit{Antwort} die Welt dem Menschen erschafft -- da setzt due Form ein, die wir \textit{Mythe} nennen wollen.\footnote{\cite{Jolles}, pp. 96-7.}
\end{singlespace}\end{quote}
E il mito stesso, risposta a una domanda, si distingue dal racconto mitico che ne è la sua forma `attualizzata' e isolata. Il mito invece è un oggetto che si fa creazione dalla sua stessa natura, l'equivalente del lavoro del contadino, che attende di essere prodotto artigianalmente.

Tuttavia Jolles sottolinea che il mito non è il frutto di una appercezione personificante e che nel mondo dello spirito non sono gli oggetti naturali a essere spiegati geologicamente, ma un giudizio personale che comporta il passaggio da \textit{mythos} a \textit{logos}.

Il mito è passaggio dalla molteplicità all'unicità, mediante il filtro del gesto linguistico che ne salda la forma in un evento unico e irripetibile.
Dunque il mito è l'elemento costante e ricorrente attualizzato nel racconto mitico, ossia nel gesto verbale che fissa un evento singolo per unificare il molteplice.

In ultima analisi Jolles distingue due tipi di mito: eziologico e analogico (o imitativo). Quest'ultimo si distingue da quello eziologico, di cui si è trattato finora, in virtù d'essere una simulazione mitica e non un mito. Ossia la creazione di un mito cosciente e derivato. Jolles si riferisce esplicitamente alle ricerche di Karl Reinhardt sulla mitopoiesi platonica\footnote{\cite{Reinhardt}, in realtà il libro tratta non solo della mitopoiesi, ma anche dell'immagine mitica e della sua correlazione con l'anima, analizzando l'opera di Platone dai dialoghi giovanili fino al \textit{Timeo}. Il mito e il \textit{logos} sono in Platone, secondo Reinhardt, in  un rapporto di reciprocità; ciò che il \textit{logos} spiega è rivelato dal mito. Il percorso segnato da Reinhardt è evolutivo e di emancipazione del mito dalla sua forma più ``decorativa'' a una dimensione filosoficamente autonoma, relativa alla capacità dell'anima di creare immagini originarie, visione comunque superata, ma affascinante e come lo ha definito Gadamer, Reinhardt ha interpretato la genesi del mito platonico come una <<cosmogonia dell'anima>> (\cite{Gadamer}, p. 165). Si veda anche \cite{Busch}.}sottolineandone il meccanismo anti-conoscenza; ossia il processo dello spirito che pur di non conoscere, si rifugia nel mito.

Lo spazio dedicato a Jolles è fondamentale per capire alcuni dibattiti guida del '900.  Egli è stato il primo ad avvertire come il mito sia un concetto molto astratto che ha bisogno, per essere compreso nella letteratura, di una attualizzazione e a problematizzare il rapporto mito-\textit{logos} in termini più generali.  Definire il mito infatti una forma semplice, anche se forse troppo schematicamente, delinea in maniera chiara un concetto che invece può apparire ancora più fumoso, e significa inoltre porre sotto ai riflettori la necessità, già citata più volte, delle definizioni.

Per tutto questo Brunel dedica un capitolo molto approfondito allo studioso olandese ed è sembrato opportuno dare un, seppur minimo, spazio alla sua trattazione che, come si è visto, ancora molto attuale.

\section{Da Jolles a Trousson}

L'analisi di Jolles da cui si è deciso di partire è centrale per comprendere il dibattito novecentesco riguardante due rapporti fondamentali che sono tra di loro consequenziali ossia tra oralità e scrittura e tra mito e letteratura, o se vogliamo tra mito e  \textit{logos}. In Jolles ci si trova davanti ancora a un rapporto di continuità tra le due nozioni di mito e di letteratura in quanto, come lungamente dimostrato, la letteratura sulla mitologia si presenta come attualizzazione del mito-forma semplice. Questa continuità sarà costantemente minata e rimessa in gioco, da una parte per la mancanza di definizioni precise, dall'altra per la presenza di un più ampio dibattito metodologico riguardante quale sia il modo <<esatto>> per analizzare i prodotti letterari. Il '900 (fino a oggi) è infatti un ``secolo di metodologie'', nel senso di elaborazione di disparate teorie della letteratura che si avvalgono di procedimenti anche non strettamente considerati letterari (quali lo strutturalismo o la psicanalisi). Non è mia intenzione tracciare un percorso sistematico del dibattito novecentesco sulla critica letteraria, tanto per motivi di spazio, quanto per motivi di interesse specifico del presente lavoro. Mi preme tuttavia fornire alcune tappe fondamentali della questione per arrivare agli studi più recenti.

Vladimir Propp (1895-1970), riprendendo e contestando parzialmente la classificazione di fiabe dei fratelli Grimm, sottolineava che in realtà queste classificazioni non tenessero né conto delle funzioni dei personaggi nelle fiabe,\footnote{\cite{Propp}, pp. 10-1. In realtà le critiche si estendono anche alle classificazioni di Wundt, Volkov e Bédier.}  né del ricco patrimonio mitico da cui le fiabe attingono, diventando così degli ingestibili e meri elenchi. Ricorda infatti Propp che i miti ancestrali sono parte di una tradizione orale antichissima; i miti scritti tanto quanto le fiabe (Propp, parla di miti preclassisti, riferendosi ai miti amcestrali) sono invece reinterpretati per giustificare il potere delle classi dominanti, e infatti proprio i contenuti dei miti ancestrali sono ribaltati nelle fiabe letterarie. Si pensi all'esempio famoso del bosco\footnote{\cite{Propp}, pp. 83ss.} che da luogo positivo, in quanto privilegiato per l'esistenza collettiva, si trasforma, tanto nei miti greci quanto nelle fiabe russe, in un luogo pericoloso, associato all'insidia e alla morte.\footnote{L'analisi del mito e della fiaba di Propp ha in realtà una più vasta origine ed espansione. Quando infatti uscì la traduzione francese di \textit{Morfologia della fiaba} si scaternò un dibattito tra Propp e Lévi-Strauss riguardante la dicotomia mito-fiaba. Tale dibattito, presente nell'edizione italiana di \textit{Morfologia della fiaba} edita da Einaudi nel 1966, sottolinea una antitetica visione della dialettica mito-fiaba che è caratteristica della seconda parte del '900 (\cite{Propp1}). Per Propp vi è una netta distinzione tra mito-verità e favola-menzogna. Egli rivendica la priorità storica del mito. D'altro canto Lévi-Strauss sosteneva che la favola si presentasse come una trasposizione attenuata di temi di cui il mito è una realizzazione amplificata, più `libera' per il fatto di non essere condizionata dal triplice criterio della coerenza logica (cfr. \cite{Jesi}, p. 28).}

Questa visione subordinante della letteratura nei confronti del mito è una delle principali linee teoriche fino agli anni '60 (e anche oltre). Nel 1939, Denis de Rougemont sosteneva già che  la letteratura fosse una forma di corruzione della purezza esoterica e sacra, una reale profanazione di un paradiso perduto del mito: <<Lorsque les mythes perdent leur caractère ésotérique  et leur fonction sacrée, ils se résolvent en littérature>>.\footnote{\cite{Rougemont}, p. 203.} Tuttavia questa corruzione non significa perdita totale, ma il mito rimane comunque come costante sottesa nelle narrazioni letterarie. Questo è stato il presupposto del libro del 1939 per indagare le <<nuove metamorfosi di Tristano>>.  La visione desacralizzata del mito nella letteratura fu accettata da molti critici della seconda metà del '900, tra i quali uno dei bersagli critici di Brunel, Raymond Trousson (1936-2013). Egli sostiene che il mito è la materia grezza che si oppone alla chiusura dell'opera letteraria, prodotto finito che fa cessare il mito. 

Questa distinzione forte tra mito e letteratura spinge dunque a distinguere tra un mito ancestrale e primordiale, e un tema o  un <<mito letterario>>, per usare le parole di di Pierre Albouy.\footnote{\cite{Albouy}, p. 9. Derivato dalla distinzione operata da Dumézil (1898-1986) tra il mito e la sua <<carrière littéraire>> (\cite{Dumezil}, p. 10).} La tematologia, nel senso etimologico di analisi di <<ciò che è dato>>\footnote{\cite{Souiller}.}, si propone proprio di indagare il mito letterario rintracciabile in testi anche diacronicamente molto distanti.

L'indirizzo storico-tematico, bisogna ammettere,  era in realtà in nuce già all'interno degli studi sul folklore e della letteratura popolare (le ricerche dei fratelli Grimm di Bolte e Polivka e di Afanasjev per esempio)\footnote{Mi riferisco in particolare a \cite{Grimm}, anche se i lavori dei fratelli Grimm in merito sono, come noto, molti di più; \cite{Bolte}; \cite{Afanasjev}, la cui edizione russa risale agli anni 1865-9.} e comparata di inzio XIX secolo, legatasi poi alla \textit{Stoffgeschichte}\footnote{Di cui emblematici prodotti sono i due dizionari di Elisabeth Frenzel, \textit{Stoffe der Weltliteratur}, e \textit{Motive der Weltliteratur} (rispettivamente \cite{Frenzel} e \cite{Frenzel1}). Sottolineo inoltre che nel 1970 l'editore Colin inaugurò la collana ``Mythes'', sotto la direzione di Pierre Brunel e Philippe Sellier e che uno dei primi lavori fu proprio una nuova edizione del primo volume di Frenzel indicato.} tedesca e poi confluita nella\textit{thématologie}.\footnote{Termine coniato da Paul Van Thieghem in uno dei primi manuali organici sulla letteratura comparata (\cite{Thiegem}), sebbene egli si dimostri molto scettico sulla metodologia e sui risultati raggiunti da questi studi; d'altro canto opinione diffusa in quegli anni, si pensi alla critica rivolta da Croce nel breve intervento su <<La critica>> nel 1903 (\cite{Croce}).}

Tuttavia fino agli anni '60 la tematologia, sia per l'eccessiva eterogeneità metodologica, sia per un rigetto generalizzato, rimane un campo secondario nella critica letteraria. A partire invece dalla seconda metà del nuovo secolo (sebbene le date siano indicative e aleatorie) si assiste grazie soprattutto ai lavori di Raymond Trousson (1936-2013) e di Harry Levin (1912-1994) a una `nuova' tematologia.\footnote{Si tratta in realtà della corrente che viene comunemente chiamata \textit{nouvelle critique} sotto il cui segno operano molti studiosi, appartenenti soprattutto alle scuole ginevrina e francese. Emblematico, per comprendere il clima di novità di questi anni, l'articolo di Trousson del 1964 \textit{Plaidoyer pour la ``Stoffgeschichte''} (\cite{Trousson1}), il quale si presenta come una lunga difesa della critica tematologica e mette un punto di inizio alle discussioni metodologiche successive. Trousson sostiene che sia necessario definire modalità e cause della continua <<palingenesi>>. (\cite{Trousson1}, p. 104).} È Trousson stesso a distinguere la critica <<tematica>>, che indaga un tema ricorrente in una singola opera, e la critica <<tematologica>>, che si cimenta nello studio tanto diacronico quanto sincronico della trasformazione di un dato tema.\footnote{Questa distinzione sarà ripresa anche da Brunel al capitolo proprio dedicato a \textit{thématique} e \textit{thématologie} in \textit{Qu'est-ce que la littérature comparée} (\cite{Brunel2}). Si veda inoltre che nel 1976 si tenne a Université Libre de Bruxelles un convegno dal nome \textit{Thématique et Thématologie} proprio per ovviare alla mancanza di distinzione netta tra i due termini. Gli atti del convegno sono stati pubblicati nel numero 43 (volume 5, 1977) della <<Revue des langues vivantes>>.} E ancora è lo stesso Trousson a distinguere dal ``tema dell'eroe'' al ``tema di situazione'', dando una spinta consistente a una classificazione maggiormente sistematica dei miti.\footnote{\cite{Trousson}, p. 36.} 

Lo studioso mette in pratica le linee teoriche proposte sul tema in \textit{Le Thème de Prométhée dans la littérature européenne}, edito nel 1964.\footnote{\cite{TroussonProm}.} Il libro riceverà numerose critche dai comparatisti contemporanei. In particolare Trousson, pur cercando di isolare le sfere di mito e di tema, tende a usare i due termini in maniera analoga, così come evidenziato sia da Albouy, che appunto propone il termine \textit{mythe littéraire}\footnote{\cite{Albouy}, p. 9: <<Il [Trousson] hésite à employer le mot \textit{mythe}, qui'il voudrait réserver au domaine reigieux et rituel qui fut le sien, à l'origine. Nous proposerons alors le terme de \textit{mythe littéraire} et nous le distinguerons du \textit{thème}. Celui-ci consiste dans l'ensemble des apparitions du personnage mythique dans le temps et l'espace littéraires envisagés>>; e ancora più avanti la definizione: <<Je définirais le \textit{mythe littéraire} come l'élaboration d'une donnée traditionelle ou archétypique, par un style propre à l'écrivain et à l'oeuvre, dégageant des significations multiples, aptes à exercer une action collective d'exaltation et de défense ou à exprimer un état d'esprit ou d\^ame spécialement complexe>> (Ivi, p. 301).} sia da Brunel, che propone un diverso schema di lettura.

La diatriba tra tema, motivo, mito e mito letterario è in realtà molto più vasta di quanto sovra citato. Molte scuole di pensiero e correnti opposte si sono susseguite nella seconda metà del '900. Fulcri centrali della critica sono tanto le sfere di influenza di temi e motivi, quanto il modo di trattare i miti; le correnti critiche principali che sono ai due capi opposti della diatriba (con molte sfumature nel mezzo in realtà) sono frutto della questione della possibilità di trattazione di un mito come un tema letterario o meno.\footnote{Per un profilo storico completo ed esaustivo sulla critica tematica e su qeusta diatriba, si può consultare un articolo di Matteo Lefèvre, edito prima su <<Allegoria>> nel 	2003 e poi ripubblicato, in versione più estesa, nel 2006 (\cite{Lefevre}).} 

Brunel proprio nel suo opporsi a Trousson propone una revisione terminologica che mi sembra interessante riassumere schematicamente:\footnote{Pubblicato per la prima volta in \cite{Brunel3} e successivamente ripubblicato nella \textit{Mythocritque}, \cite{Brunel1}, pp. 27-37.}
\vspace{1cm}
\begin{center}
\begin{tabular}{| p{4cm} | p{4cm} | p{4cm} |}\hline
Trousson&Tema = Pometeo&Motivo = Rivolta\\\hline
Brunel&Tema = rivolta&Motivo = elemento del mito e non un insieme\\
&Tipo = Prometeo (ribelle)&\\\hline
\end{tabular}
\end{center}
\vspace{1cm}

L'inversione di prospettiva operata da Brunel è fondamentale per capire l'impostazione del suo lavoro. Egli infatti critica l'impostazione troussoniana in quanto tendente a ridurre il mito a una mera ``situazione'' particolare, o al nome dell'eroe, minando così all'impostazione iniziale del lavoro. Per chiarire questa sua posizione Brunel ricorre al mito di Oreste, sostenendo che esistono sia parricidi che non si chiamano Oreste, sia degli Oreste che non sono parricidi (quello di Omero per esempio).\footnote{\textit{Ibidem}.} E in quest'esempio la critica copre entrambe le divisioni operate da Trousson, o si sceglie di studiare un mito in tutte le sue forme, o si sceglie di studiare un tema in tutte le sue ricadute.

Per quanto riguarda il ``motivo'', Brunel sottolinea che, al di là del dibattito di fine '900 al riguardo, esso deve essere considerato alla stregua di un elemento del mito che non può in nessun modo esaurire la totalità del mito stesso. Che sia infatti un elemento del mito variabile (come lo considera Lévi-Strauss)\footnote{La concezione di Lévi-Strauss (esposta in \cite{Levi1}, \textit{passim} e \cite{Levi2}, p. 240) è di vedere il mito alla stregua di un fenomeno fonetico; da una parte egli identifica dei <<mitemi>> ossia delle parti distintive minime, e dall'altra c'è il prodotto combinato da essi, l'equivalente del suono prodotto da una parola.} o che sia un elemento iterativo (secondo le parole di Génot), non può che essere considerato comunque un mero elemento minimo.

\section{La tradizione <<emblematica>> e la \textit{Mythocritique} di Brunel}
\lhead{\textit{1.\space 4.  LA TRADIZIONE <<EMBLEMATICA>>}}
La \textit{Mythocritique} di Brunel è da considerare uno dei testi chiave della mitocritica moderna. Alla stregua di come si è deciso di procedere per Jolles, è necessario dedicare il dovuto spazio a questo testo che ha dato una svolta importante agli studi tematologici e tematici. Esso infatti si presenta come il punto di incontro dei filoni critici della \textit{Nouvelle critique} sviluppatasi nella seconda metà del '900, e come snodo fondamentale per la critica più moderna. Inoltre, è, insieme a Trousson, uno dei fulcri attorno ai quali ruota il metodo applicato alla tematologia, metodo necessario per l'analisi ``emblematica''.

\subsection{Che cos'è la <<mitocritica>>?}
Il termine ``mitocritica'' è neoconiazione di Gilbert Durand. Esso nasce a specchio distorto della <<psicocritica>>, inaugurata da Charles Mauron, per indicare una critica letteraria che indaghi l'inconscio dello scrittore sulla base di ``metafore ossessive'' ricorrenti e involontarie.\footnote{Cfr. \cite{Mauron}, p. 68.} Mediante lo studio di queste metafore ricorrenti si può, secondo Mauron, risalire a un <<mito personale>> dell'autore.

Durand, in \textit{Les Structures anthropologiques de l'imaginaire} --il primo di una lunga serie di libri sul tema-- elabora, in negativo, a partire dall'enunciazione di Mauron, il termine <<mitocritica>>.\footnote{L'idea è gia presente in nuce in \textit{Les Structures anthropologiques de l'imaginaire} (\cite{Durand}) tuttavia l'elaborazione terminologica è successiva in quanto atta a contrapporsi al termine di Mauron (\cite{Durand1}). Dice infatti Brunel che l'idea della <<mitocritica>> è presente in Durand prima ancora della sua formulazione (\cite{Brunel1}, p. 39).} Egli contesta a Mauron, basandosi su un'idea di mito superiore e ``onnipotente'' che non è riducibile a un mito personale. La ``mythodologie'' di Durand è essenzialmente junghiana e apertamente critica dello strutturalismo letterario. Brunel contesta parzialmente questa posizione e propone un'idea di mitocritica che ricade completamente nella scia del filone tematologico. La <<mitocritica>> per Brunel, dunque, è un filone di analisi critica della letteratura che ha poco, se non nulla, a che fare con l'idea alla base della quale il termine è nato. 
È Brunel stesso, d'altro canto, a distaccarsi tanto dal concetto di mitocritica durandiano tanto dalla <<mythanalise>> di Denis de Rougemont.

\subsection{Il metodo di Brunel: funzioni e struttura del mito}
Nella prefazione del \textit{Dictionnaire des mythes littéraire}, Pierre Brunel decide di non definire il termine ``mito'' ma di definire il mito in base alle sue funzioni:
\begin{enumerate}
\item Il mito racconta, perché è narrazionne
\item Il mito spiega, in quanto è eziologico
\item Il mito rivela, perché, secondo le parole di Eliade, tutte le mitologie sono anche <<ontofanie>>

Egli contesta la visione del mito come forma di degradazione di un mito primordiale, sostenendo apertamente le posizioni di Regis Boyer (e non solo), perché il contatto che il singolo ha col mito è puramente letterario. La conclusione di Brunel è che il mito debba essere studiato come un tema, esattamente come già in nuce, sebbene con delle prospettive differenti, in Trousson.\footnote{\cite{Brunel4}, p. 11.}

Sottolinea inoltre che sul mito etno-religioso, quando si fa letteratura, operano (cfr. § 1.1) due forze contrastanti; da una parte scompaiono tre elementi: il mito letterario non è fondatore; le opere che illustra sono segnate in principio; e il mito letterario non è accettato come verità.\footnote{Ivi, p. 13.} D'altro canto le loro carratteristiche comuni\footnote{Questa formulazione Brunel la riprende da Sellier (\cite{Sellier}, \textit{passim}).} sono la saturazione simbolica, l'organizzazione serrata e <<l'illuminazione metafisica>>\footnote{\cite{Brunel4} p. 13.} (ossia spiega, fa chiarezza su qualcosa).



Questa dialettica di perdita e acquisto che si istaura nella letteratura mitica fa ragionare sul contenuto effettivo dei miti letterari, e fa comprendere come questi siano passibili di uno studio tematico. Esso infatti ha la capacità di nascere e rinascere in epoche diverse e sotto diversi aspetti (è polivalente e polisemico, come tutti i temi letterari). Il mito, con tutte le sue funzioni e il suo contenuto deve essre considerato un linguaggio preesistente al testo, ma che è diffuso nel testo stesso.\footnote{\cite{Brunel1}, p. 61}

Brunel infatti propone di leggere come analoghe le strutture del mito letterario e di un testo qualsiasi delle letteratura. Partendo infatti dalle analisi strutturali del testo di Julia Kristeva, lo studioso sostiene che ciò che interessa alla mitocritica è la comprensione delle analogie tra testo e mito. Questo non solo per la presenza di forze antagoniste e contraddizioni all'interno di entrambi, ma anche per l'armonia tra gli elementi la cui modificazione fa crollare un'impalcatura su cui si basano le strutture di entrambi.

\end{enumerate}

\subsection{Il metodo di Brunel: emergenza, flessibilità e irradiazione}
Pur non cadendo mai nel dogmatismo, Brunel fornisce uno schema teorico accurato e attuale, come ben sottolinea Véronique Gély nella postfazione dell'edizione italiana di \textit{Mythocritique}.\footnote{\cite{Brunel}, p. 83} Questo metodo di analisi letteraria, che parte dai presupposti già citati nel paragrafo precedente, aveva il proposito esplicito di fondare un metodo preciso per la mitocritica. 
Le tappe sono tre e converrà che siano brevemente riassunte.\footnote{\cite{Brunel1}, pp. 72-86.}

\textsc{\'{E}mergence}: Esame delle occorrenze mitiche nel testo, nominare un oggetto significa automaticamente evocarlo esplicitamente. 

\textsc{Flexiblité}: Il mito può essere rievocato in situazioni molto diverse le une dalle altre, è necessario dunque capire gli adattamenti. Infatti il mito è ambiguo, polisemico e ha la peculiarità di parlare per enigmi o comunque a lasciar loro spazio, motivo per cui è anche aperto a molteplici interpretazioni.

\textsc{Irradiation}: spesso i miti sono letti come sopravvivenze nostalgiche di un passato antico e irrecuperabile e la mitocritica si oppone radicalmente  a questo <<scepticisme dédaigneux>>. L'elemento mitico deve avere un potere di irradiazione.

Nota bene Cettina Rizzo, nella premessa alla traduzione italiana della \textit{Mythocritique}, che l'approccio critico di Brunel porta a trattare i miti come fenomeni sempre nuovi, senza imbrigliarli in rigorose regole generali, ma tenendo conto delle relazioni sincroniche.\footnote{\cite{Brunel}, pp. V-VI.} 

\section{La tradizione <<archetipica>> e gli archetipi dell'immaginario}
\lhead{\textit{1.\space 5. LA TRADIZIONE <<ARCHETIPICA>>}}
Descrivendo metodo dell'\textit{émergence} Brunel dice che Durand, nell'individuare un non esplicitamente citato Eracle nella \textit{Chartreuse de Parme} per alcuni tratti comuni con il personaggio (tuttavia ravvisabili anche in Teseo e Poseidone), si lascia trasportare dalla teoria archetipica, abbandonando la mitocritica: 
\begin{quote}\begin{singlespace}
\footnotesize{
La mythocritique est ici débordée par ce qu'il serait plus juste d'appeller une archétypocritique, ou du moins la recherche de structures qui peuvent \^etre communes à plusieurs mythes sans en caractériser aucun>>.\footnote{\cite{Brunel1}, p. 74.} 
}\end{singlespace}
\end{quote}

A partire da quest'affermazione di Brunel, il cui giudizio negativo sull'operato di Durand, in realtà egli riesce a dare una chiave di lettura positiva e specifica della critica archetipica. In quest'ultima sezione infatti è premura di analizzare brevemente cosa sia la critica archetipica e quale sia la metodologia da applicare nel caso in cui si voglia risalire all'archetipo di un mito, soggiacente ad alcune narrazioni letterarie.

La nozione di archetipo infatti è di grande interesse per la seconda parte della presente ricerca. Il terzo capitolo infatti utilizza tale nozione (e tale filone critico) per analizzare alcuni aspetti peculiari e ricadute del mito di Endimione nel Rinascimento.

\subsection{L'archetipo e la letteratura}

Grazie agli studi di James Frazer (1854-1941) e alle teorie di Carl Gustav Jung (1875-1961), la critica archetipica sposta l'attenzione della ricerca  letteraria sui ``motivi''\footnote{Si è già accennato alla diatriba terminologica, in questo caso si è deciso di indicare ``motivi'' per tener fede allo scritto di Jung, che identifica gli archetipi letterari nei motivi.} ricorrenti che formano l'esistenza.

L'archetipo è, secondo la definizione junghiana, un contenuto psichico non ancora sottoposto all'elaborazione conscia, ma che si presenta come un dato psichico immediato.\footnote{\cite{Jung}, p. 15.}  La riflessione sugli archetipi è il frutto di una ricerca sull'inconscio collettivo, la cui natura è ingenerata e universale, e, di conseguenza, non derivante da nessuna esperienza personale specifica,\footnote{Ivi, pp. 1-2.} ma da un'esperienza ancestrale comune che è entrata a far parte del cervello di tutti. Jung distingue in questa maniera l'inconscio collettivo dal <<pers\"onliche Unbewusste>>\footnote{Ivi, p. 13.} che invece è il frutto della rimozione di esperienze personali che rimangono incoscie, e la cui base sono i <<gef\"uhlsbetonten Komplexe>>\footnote{\textit{Ibidem}.} (agglomerati di emozioni). Al contrario l'esperienza collettiva si fonda, proprio in virtù della sua universalità, su delle forme collettive \textit{a priori} a cui Jung  dà il nome di <<archetipi>>.\footnote{La nozione di archetipo viene approfondita da Jung anche all'interno di \textit{Über die Beziehung der analytischen Psychologie zum dichterischen Kunstwerk} nel quale egli definisce l'archetipo una figura o processo che si ripete nel corso della storia ogni qualvolta la fantasia si eserciti liberamente (\cite{Jung1}, \textit{passim}).}
Il soggetto dunque, alla nascita, presenta una psiche già performata su delle rappresentazioni comuni (quelle dell'inconscio collettivo) eterogenee che però accomunano tutte le esperienze.  Il fattore che rende indispensabili gli archetipi è quello per cui, in virtù di questa comunanza tutta umana, le esperienze sono rese comprensibili e soprattutto confrontabili. Gli archetipi dunque non sono solo delle forme, ma anche delle <<possibilità innate di rappresentazioni>>.\footnote{\cite{Jung1}, p. 93.}

Nella letteratura possono essere considerati archetipi tutti quei motivi che costantemente danno forma all'esistenza umana. Ossia non riducibili biograficamente, storicamente, socialmente, ma che invece rimandano a un'immagine primordiale (forma, modello, addirittura idea nel senso platonica del termine) che può essere ricondotta a una costante archetipica (si potrebbe dire, a un universale dell'esistenza umana).

Autori come Maud Bodkin (1875-1967), Joseph Campbell (1904-1987),  Francis Fergusson (1904-1986) e Northrop Frye (1912-1991) hanno interpretato i testi letterari secondo questa impostazione, e i loro studi, in particolore quelli di Frye, sono tra i più esaustivi in materia di critica archetipica.\footnote{Mi riferisco in particolare a \cite{Bodkin} e \cite{Frye}.}

Essi interpretarono una vasta gamma di testi letterari alla luce delle somiglianze che questi ultimi presentavano con i personaggi, le narrazioni o le situazioni mitologiche, attribuendo al concetto di archetipo definizioni diverse e nel contempo cercarono di spingersi oltre lo studio della presenza e della funzione degli archetipi nel testo letterario per interrogarsi sulla più complessa questione della natura e dell’origine delle forme archetipiche stesse.

 In \textit{Archetypal Patterns in Poetry. Psychological Studies of Imagination}, Bodkin applica alla tragedia il concetto di archetipi  e si prefissa di capire se si possano indivduare dei <<themes having a particular form or pattern which persists amid variation from age to age>> e la risposta è chiaramente positiva.\footnote{\cite{Bodkin}, p. 4. La formulazione di Bodkin muove parallelamente dalla nozione di archetipo junghiano e dall'intuizione di Gilbert Murray riguardo Amleto ed Oreste su una effettiva ``vibrazione'' comune. Murray è infatti esponenente della ``cerchia'' antropologica dei Cambridge Ritualists (pur insegnando a Oxford) che hanno inaugurato lo studio sulle spiegazioni di mito e rito. Questi autori cambiarono radicalmente la prospettiva dello studio dei testi classici; uno degli esempi è la concezione del ciclo stagionale come idea di base del dramma e della poesia, ossia la dialettica tra forza vitale (vittoriosa) e morte.}




Northrop Frye, con \textit{Anatomy of Criticism. Four Essays},\footcite{Frye} scrive il primo manifesto programmatico della ricerca mitico-archetipica. Frye infatti si propone di sistematizzare e di applicare la critica archetipica alla letteratura. In realtà egli si spinge ancora più in là nella definizione, sostenendo che l'archetipo, in quanto simbolo che permette la connessione tra due o più poemi, rende possibile l'unificazione e l'integrazione dell'esperienza letteraria, che, in buona sostanza, si basa tutta su archetipi letterari. Il presupposto di Frye è che mitologia e letteratura rispondano alle stesse aspirazioni e, di conseguenza, abbiano due modi coincidenti di rappresentazioni; in particolare egli si riferisce alla capacità di entrambe di poter effettivamente rappresentare la riconciliazione del dualismo uomo-natura. La letteratura rappresenta dunque la faccia più sofisticata della mitologia.

Sottolinea inoltre Frye che, in quanto organizzazioni simboliche, tanto il mito, quanto la letteratura rispondano a una funzione di ricorrenza, o riproposizione.\footnote{\cite{Frye}, p. 346.} E ripropongono nient'altro che dei ritmi naturali e ciclici.

La metafora più pregnante utilizzata da Frye è con la matematica. Matematica e letteratura sono due linguaggi per descrivere l'universo,\footnote{Ivi, p. 354.} ed è per questo che lo studio della letteratura permette di risalire alla forma culturale universale che non è solo del passato, ma del presente. 
Ed è per questo che il lettore, quanto il critico, quanto uno scrittore, compiono leggendo un <<rivoluzionario atto di coscienza>>\footnote{Ivi, p. 344.} perché non fanno altro che trasporre in un nuovo contesto una vecchia funzione. Lasciando parlare dunque lo stesso studioso:
\begin{quote}
The culture of the past is not only the memory of mankind, but our own buried life, and the study of it leads to a discovery in which we see, not our past lives, but the total cultural form of our present. It is not only the poet but his reader who is subject to the obligation to ``make it new''.\footnote{Ivi, p. 346.}
\end{quote}

Sarà dunque compito ultimo del critico estrapolare gli archetipi sottesi per compiere un atto di coscienza nuovo e fornirlo ai suoi lettori.

\section{Un punto sul metodo}

Questa rassegna parzialemente storica, solo parzialmente esaustiva, è stata necessaria per poter incanalare la ricerca nel campo di studi della mitocritica. Abbiamo iniziato il capitolo, sottolineando come fosse necessario infatti rendere conto di alcuni filoni passati per poter restringere il campo della selezione del vasto materiale a disposizione sul mito di Endimione. Ci si interesserà dunque delle due tradizioni emblematica e archetipica, ristretto il campo al solo '500. Un'analisi più estesa richiederebbe molto più tempo e molto più lavoro, rispetto alla modestia del presente. 

I metodi che sono stati scelti per questa ricerca sono due: un metodo tematico, che tratta il mito come tema letterario, nel caso in cui gli autori citino espressamente il mito di Endimione; e un metodo archetipico, nella presenza sottesa di un Endimione-archetipo che si presenta come una chiave di lettura di alcuni testi neoplatonici rinascimentali. Nel primo caso trattare Endimione da tema letterario servirà per comprendere alcuni aspetti perculiari delle diffusione del mito RInascimento. Un mito che è stato richiamato alla memoria da un mondo lontano e da un ruolo minore, subendo una forte emancipazione letteraria nel '500. Nel secondo caso, la tradizione di Endimione si estende molto al di fuori dei suoi limiti intrinseci, e si manifesta come paradigma di una <<bella morte>>, che sarà analizzata nel terzo e ultimo capitolo.

\chapter{Il mito di ed Endimione, l'approdo al Rinascimento}
\rhead{\textit{2. IL MITO DI ENDIMIONE}}
Di ricerche specifiche sul mito di Endimione ne esistono relativamente poche.  Il mito infatti è stato analizzato soprattutto nel suo aspetto figurativo, in quanto spesse volte rappresentato sui sarcofagi in un periodo di tempo che abbraccia dal III secolo a. C.  al IV d. C., con ricadute nell'era cristiana. Emblematici al riguardo gli studi di Sichtermann, la voce "Endymion" nel LIMC curata da Gabelmann, con accurata bibliografia, il libro di Koortbojian su Adone ed Endimione.\footnote{E si vedano anche DNP 3 (1997) 1027 (Tanja Scheer); DNP Suppl. 5 (2008) 253-7 (Marc F\"ocking), LIT. 257; \cite{Reinhardt} pp. 76-77 e \textit{passim}. \cite{Hans}, pp. 237-42.}

Si aggiunge a questi lavori un interessante testo di Francesco Gandolfo, spesso bistrattato, in quanto accusato di voli pindarici sulla ricostruzionte dell'iconografia endimionica, e in particolare della rappresentazione del mito da parte di Cima da Conegliano (cfr. Appendice 1). 

Lavoro invece più interessante quello di Natalia Agapiou che tratta il mito di Endimione con il metodo <<brunelliano>>, applicandolo in particolare al '500.  Il testo è una buona raccolta di fonti tanto letterarie, quanto iconografiche e permette di formarsi un'idea della diffusione del mito in Italia, Inghilterra e Francia tra il 1510-1647: dall'Endimione di Cima da Conegliano all'Endimione di Guercino. I filoni tematici sono più di uno, tuttavia forse trattati in maniera poco organica. Si sente dunque ancora il bisogno di un'analisi più completa de

\section{La storia e le fonti greche del mito}
\subsection{Le tradizioni orientale e la tradizione elidica}
La vulgata del mito di Endimione racconta di un amore notturno e segreto con la dea Luna, la quale decide di donargli eterna giovinezza e sonno per poter godere dei suoi amplessi. Una storia così intrinsecamente romantica non poteva che alimentare la fantasia di numerosi poeti che ne hanno fatto molto spesso un paradigma di amore assoluto. Tuttavia la storia di questo mito non nasce così.

La tradizione del mito sembra essere bipartita, due sono infatti i centri principali di propulsione, l'uno non comunicante con l'altro: una tradizione che ha come capostipite cronologico Saffo (fr. 199V), legata maggiormente al mondo orientale e alla Caria, e una tradizione che Agapiou chiama <<elladica>>,\footnote{\cite{Agapiou}, p. 19.} ma che sarebbe più corretto definire ``elèa'', che invece deve la sua origine a Esiodo.
Endimione menzionato per la prima volta da Esiodo stesso, nel \textit{Catalogo delle donne}, non inserito in una cornice selenica, ma legato all'Elide:
\begin{quote}\begin{singlespace}
\footnotesize{
\begin{otherlanguage}{greek} 
[αὐτὰρ Ἀεθλίοο κρα]\Dot{τ}ερ\Dot{ὸ}ν μένος ἀντιθέοιο\\
\Dot{ε}[ὐειδέα Καλύκην θα]λ\Dot{ε}ρὴν ποιήσατ’ ἄκοιτιν·\\
\Dot{ἣ} [δ’ ἔτεκ’ Ἐνδυμίωνα] φ\Dot{ί}λ\Dot{ο}ν μακάρεσσι θεοῖσι·\\
\lbrk τὸν δὲ Ζεὺς τίμησ]\Dot{ε}, \Dot{π}\Dot{ε}\Dot{ρ}\Dot{ι}\Dot{σ}\Dot{σ}\Dot{ὰ} δὲ δῶρα ἔδωκεν,\\
ἳν δ’ α>uτ\~wι θανάτου ταμίης καὶ γήραος ἦεν.\\
\lbrk τοῦ δ’ ἦν Αἰτωλός· το]\Dot{ῦ} \Dots{δ}’ \Dot{α}\Dot{ὖ} \Dot{Κ}\Dot{α}\Dot{λ}\Dot{υ}\Dot{δ}ὼν γ\Dot{έ}ν\Dot{ε}θ’ υἱός\\
Π[λευρών τ’ αἰχμητ]\Dot{ή}\Dot{ς}, \Dot{ἐ}\Dot{π}\Dot{ι}\Dot{ε}\Dot{ί}\Dot{κ}\Dot{ε}λος ἀθανάτοισιν,\\
\lbrk \hspace{2cm}ὃς           ]. \Dot{Ἀγ}ήνορα γείνατο παῖδα·\end{otherlanguage}}\footnote{Hes. Fr. 10a MW = 5 H.  \nocite{EsiodoHirsch} \nocite{EsiodoWest}}
\end{singlespace}\end{quote}

Endimione, figlio di Calice ed Etlio, e padre dell'ecista Etolo, riceve da Zeus, dispensatore di morte e di vecchiaia, doni eccezionali. Precisa uno scolio al quarto libro di Apollonio Rodio,\footnote{Schol. Ap. Rhod. 4. 58 Wendel, pp. 264.16-265.1 \nocite{WendelAp}.} citando un  altro passo di Esiodo, estratto invece dalle \textit{Grandi Eoie} che Endimione non ricevette un regalo da Zeus, ma una punizione. Il giovane mortale infatti, portato nell'Olimpo da Zeus, si invaghì di Era, apparsa sotto forma di una nuvola e  Zeus lo punì gettandolo nell'Ade. Nella versione esiodea la Luna non è menzionata né come attrice né come forza passiva (come invece variamente interpretata), e colloca il \textit{background} mitico in Elide.

Endimione fa del mortale un personaggio storicamente attestato, oltre che elevarlo a rango di eroe-ecista, dalla illustre schiatta. Dello stesso tenore la teorizzazione di Pausania che cerca di conciliare le due tradizioni, Paus. Perieg. 5.1.3-5), sebbene si crei una certa confusione.\footnote{\cite{Graf}, in particolare si veda il sottocapitolo\textit{The Mythology of Elis According to Pausanias}, alle pp. 213-8.} 

\begin{quote}\begin{singlespace}\footnotesize{
τοὺς Ἠλείους ἴσμεν ἐκ Καλυδῶνος διαβεβηκότας καὶ Αἰτωλίας τῆς ἄλλης: τὰ δὲ ἔτι παλαιότερα ἐς αὐτοὺς τοιάδε εὕρισκον. βασιλεῦσαι πρῶτον ἐν τῇ γῇ ταύτῃ λέγουσιν Ἀέθλιον, παῖδα δὲ αὐτὸν Διός τε εἶναι καὶ Πρωτογενείας τῆς Δευκαλίωνος, Ἀεθλίου δὲ Ἐνδυμίωνα γενέσθαι: [4] τούτου τοῦ Ἐνδυμίωνος Σελήνην φασὶν ἐρασθῆναι, καὶ ὡς θυγατέρες αὐτῷ γένοιντο ἐκ τῆς θεοῦ πεντήκοντα. οἱ δὲ δὴ μᾶλλόν τι εἰκότα λέγοντες Ἐνδυμίωνι λαβόντι Ἀστεροδίαν γυναῖκα—οἱ δὲ τὴν Ἰτώνου τοῦ Ἀμφικτύονος Χρομίαν, ἄλλοι δὲ Ὑπερίππην τὴν Ἀρκάδος—, γενέσθαι δ᾽ οὖν φασιν αὐτῷ Παίονα καὶ Ἐπειόν τε καὶ Αἰτωλὸν καὶ θυγατέρα ἐπ᾽ αὐτοῖς Εὐρυκύδαν. ἔθηκε δὲ καὶ ἐν Ὀλυμπίᾳ δρόμου τοῖς παισὶν ἀγῶνα Ἐνδυμίων ὑπὲρ τῆς ἀρχῆς, καὶ ἐνίκησε καὶ ἔσχε τὴν βασιλείαν Ἐπειός: καὶ Ἐπειοὶ πρῶτον τότε ὧν ἦρχεν ὠνομάσθησαν. [5] τῶν δὲ ἀδελφῶν οἱ τὸν μὲν καταμεῖναί φασιν αὐτοῦ, Παίονα δὲ ἀχθόμενον τῇ ἥσσῃ φυγεῖν ὡς πορρωτάτω, καὶ τὴν ὑπὲρ Ἀξιοῦ ποταμοῦ χώραν ἀπ᾽ αὐτοῦ Παιονίαν ὀνομασθῆναι. τὰ δὲ ἐς τὴν Ἐνδυμίωνος τελευτὴν οὐ κατὰ τὰ αὐτὰ Ἡρακλεῶταί τε οἱ πρὸς Μιλήτῳ καὶ Ἠλεῖοι λέγουσιν, ἀλλὰ Ἠλεῖοι μὲν ἀποφαίνουσιν Ἐνδυμίωνος μνῆμα, Ἡρακλεῶται δὲ ἐς Λάτμον τὸ ὄρος ἀποχωρῆσαί φασιν αὐτὸν καὶ τιμὴν αὐτῷ νέμουσι, καὶ ἄδυτον Ἐνδυμίωνός ἐστιν ἐν τῷ Λάτμῳ.}\end{singlespace}\nocite{Pausania}
\end{quote}
Pausania non dà credito al legame tra Endimione e Selene, in quanto deve dare una cosistenza storica alla genealogia elèa, ma sostiene che Endimione prese in moglie Asterodea da cui generò tre figli maschi e una femmina, illustri antenati anch'essi di Peoni ed Epei (come il padre eponimo degli Etoli). Pausania razionalizza e cerca di conciliare le due versioni del mito (par. 5.1.5), sostenendo che Eraclei ed Elei discutessero in merito all'origine di Endimione e che gli Eraclei rivendicassero l'autoctonia del mortale di cui la città ai piedi del Latmo manterrebbe un santuario segreto. Sembra comunque che Pausania stesso, anche per la presenza di una tomba di Endimione che mal si accorda con la versione orientale.

Questo Endimione è lo stesso che avrebbe istituito una gara di corsa a Olimpia per determinare il nuovo re discendente (par. 5.1.4). Il legame Olimpia - Endimione, come ribadito anche in un altro passo della \textit{Periegesi}\footnote{Paus. 5.8.2: <<Ἐνδυμίων δὲ ὁ Ἀεθλίου Κλύμενόν τε ἔπαυσε τῆς ἀρχῆς καὶ δρόμου τοῖς υἱοῖς ἆθλα ἐν Ὀλυμπίᾳ τὴν βασιλείαν ἔθηκε>>.} è volto a sottolineare. 
Sostiene comunque a ragione Agapiou che il nome di Endimione nel Peloponneso rimarrà legato agli Elèi,\footnote{\cite{Agapiou}, p. 20.}, così come sottolineato anche dal geografo del VI secolo d. C. Stefano di Bisanzio che sostiene che gli Elèi si facessoro chiamare anche \begin{otherlanguage}{greek} >Endumiwn'adai \end{otherlanguage} per il nome del loro capostipite.\footnote{Meineke, 272. [controlla citazione]}

Le conclusioni di Pausania (110-80 d. C.) testimoniano una tradizione molto dubbia, che il periegeta tenta di razionalizzare. Esistono almeno due versioni contrastanti del mito di Endimione, e una delle due è legata a Selene. All'altezza in cui scrive Pausania tuttavia, come vedremo nel prossimo paragrafo, la situazione di Endimione è maggiormente legata alla mitografia selenica, i sarcofagi, tanto quanto la tradizione romana, tendono infatti a prediligere la versione amorosa del mito. Le motivazioni della predilezione sono due: da una parte a livello letterario è più interessante il mito di una dea che si innamora di un mortale, della genealogia di un ecista greco; dall'altra la connotazione fortemente escatologica del mito di Endimione, dovuta al suo sonno immortale e al collegamento diretto tra i due fratelli Ipno e Thanatos, ne fa un perfetto soggetto funerario.
 
L'Endimione ecista, dalle nobili discendenze, è una situazione diametralmente opposta rispetto alla condizione di paredro mortale della dea Selene, di umili origini, ma di bellezza `sovrannaturale'. Non è ben chiaro come questi due miti fossero effettivamente correlati. Un'iscrizione che riporta un rudimentale partenio, rinvenuta a Eraclea, conservata al Louvre, e pubblicata da Dain nel 1933, con il commento di Wilamowitz,\footnote{\cite{Dain}, pp. 67-73, n. 60. Iscrizione già studiata da Bernard  Haussoullier, la cui pubblicazione fu stroncata dalla morte dello studioso nel 1926, ma di cui, sebbene parzialmente per mancanza delle carte, Dain tenta di dar resoconto. Si deve proprio ad Hassouiller l'identificazione del testo con un partenio.} dà una spia sebbe non chiara del legame tra le due tradizioni. Nell'iscrizione, mutila per metà, si parla di una festa dedicata a Endimione, fondatore della città (di Eraclea) e addormentato nell'antro, e a due dee, probabilmente Selene e Atena o una dea locale. La festa celebra la fine delle grandi sofferenze (forse un terribile inverno). Il collegamento tra Eraclea al Latmo (in Caria) e l'Elide è stato portato avanti da Louis Robert. Commentando un decreto etolo del III secolo a. C. in cui viene accordata a una Eraclea (le Eraclea in Grecia sono sei) l'isopolitismo in virtù di un'antica parentela, sostiene la discendenza degli Etoli dai Cari.\footnote{\cite{Robert}, pp. 477-90. Alle pp. 477-80 è riassunta la diatriba sull'identificazione di Eraclea, e sono indicate tutte le posizioni in merito degli sudiosi precedenti a Robert.} D'altro canto questa discendenza sarebbe giustificata dalla trafila Etolo-Endimione, testimoniata da Esiodo, Pausania et alii.

La curiosità della connessione delle due tradizioni del mito, che ritengo rimanga una curiosità, non giustifica l'assenza totale della storia di Endimione e  Selene nella parte elèa ed etolica della Grecia. Mi sembra abbastanza evidente che, allo stato dei fatti, non sia ancora possibile costruire un ponte sicuro tra le due versioni mitiche, che dunque continuerò a trattare come distinte.\footnote{Udo Reinhardt ovvia alla distinzione dicendo: <<urspr\"unglich also wohl ein kleinasiatischer Heros, sp\"ater in griechischer Genealogie K\"onig von Elis.>> \cite{Reinhardt}, p. 76.}  È ovviamente chiaro che un culto collegato a Selene nella penisola elladica sarebbe effettivamente peculiare.  Selene è infatti divinità ben poco nota all'occidente greco e dalle origini molto oscure. Sarà infatti presto assimilata più spesso ad Artemide (in quanto sorella di Apollo-Sole),  che ne assorbirà tutte le funzioni e le virtù, ma anche a Eileithyia, Ecate\footnote{Identificazione antica, cfr. Hymn. Orph. II, 8. \textit{In Hecaten}.} e Persefone.\footnote{A Roma sarà Cinzia, Luna o Lucina. Interessante l'identificazione con Apoc. 12.1 proposta da Udo Reinhardt (\cite{Reinhardt}, p. 195). } Si pensi in particolare che Artemide a Efeso era legata al culto della dea Notte\footnote{Cfr. Paus. 10.38.3-6. A Efeso, nell'Artemision.} e infatti proprio in questa zona (in tutto il Peloponneso) venivano celebrate delle feste notturne, le Oribasies di Solmissos e pannychie.\footnote{\cite{Ramnoux}} e Artemide viene chiama <<nyctelia>>

Legata sin dall'antichità ai boschi e ai monti probabilmente per l'accessibilità all'osservazione delle fasi lunari dalle zone montane, \footnote{Cfr. \cite{Pestalozza}, p. 350 e si veda anche \cite{PestalozzaHec}.} è infatti chiamata spesso \begin{otherlanguage}{greek}>ida'ia, >ore'ia\end{otherlanguage}.\footnote{\cite{Pestalozza}, pp. 349.} Non suonerà dunque dissonante trovare un \begin{otherlanguage}{greek}tò t\~hc Sel'hnhc >'oroc \end{otherlanguage}nei Geografica di Tolomeo, e che il mito dell'amato dalla Luna sia ambientato su un monte, il Latmo, di Caria.
 Esiodo (Es. Th. 371-4) dice essere Selene figlia di Iperione e Thea, e sorella di Elio ed Eos.\footnote{Cfr.<<\begin{otherlanguage}{greek}θεία δ᾽ Ἠέλιόν τε μέγαν λαμπράν τε Σελήνην/Ἠῶ θ᾽, ἣ πάντεσσιν ἐπιχθονίοισι φαείνει/ἀθανάτοις τε θεοῖσι, τοὶ οὐρανὸν εὐρὺν ἔχουσι,/γείναθ᾽ ὑποδμηθεῖσ᾽ Ὑπερίονος ἐν φιλότητι\end{otherlanguage}>>.} Dello stesso tenore sono le testimonianze di Acusilao di Argo, Ferecide, Pisandro e Nicandro, come testimoniato dal sovracitato scolio ad Apollonio Rodio:
\begin{quote}\begin{singlespace}
\footnotesize{
SCHOL. APOLL. RHOD. IV 57: \begin{otherlanguage}{greek} Λάτμος ὄρος Καρίας, ἔνθα
ἐστὶν ἄντρον, ἐν ὧι διέτριβεν Ἐνδυμίων, ἔστι δὲ καὶ πόλις ἡ λεχθεῖσα
Ἡράκλεια. τὸν δὲ Ἐνδυμίωνα Ἡσίοδος  μὲν Ἀεθλίου τοῦ Διὸς καὶ 
Καλύκης παῖδα λέγει, παρὰ Διὸς εἰληφότα τὸ δῶρον ἳν αὐτῶι ταμίαν
εἶναι θανάτου, ὅτε θέλοι ὀλέσθαι· καὶ Πείσανδρος \end{otherlanguage} (om. Ki)\begin{otherlanguage}{greek} καὶ Ἀκουσίλαος\end{otherlanguage}   (5)
\begin{otherlanguage}{greek} καὶ Φερεκύδης \end{otherlanguage} (3 F 121)  καὶ Νίκανδρος ἐν δευτέρωι \begin{otherlanguage}{greek}Αἰτωλικῶν  \end{otherlanguage} (F 6 Schn.) \begin{otherlanguage}{greek} καὶ Θεόπομπος ὁ ἐποποιός.\end{otherlanguage}}
\footnote{Tradizione seguita anche da Eforo, citato in Strabone X, 3-2Fr. Gr. Hist. 2a, 70, F, 122a.}\end{singlespace}\end{quote} 

Lo scolio ad Apollonio, così anche lo pseudo Apollodoro della \textit{Bilblioteca}\footnote{Ap. Bibl. 1 56. \nocite{Wagner}}, uniscono, riportando acriticamente, i due filoni della tradizione, dando maggiore credito alla tradizione più storica, mettendo il focus sul ruolo ecista di Elide, trasportatore dalla Tessaglia gli Eoli. 

Fatto sta che Endimione già nel III secolo a. C. risulta intimamente legato al monte Latmo. Un oracolo pitico rivolto ai Magnesi, suggerendo un intinerario verso la Caria, dice loro di andare <<\begin{otherlanguage}{greek}>Apenant'ion >Endum'iwnoc\end{otherlanguage}>>,\footnote{\cite{Robert}, p. 481 n. 32.} testimoniando un forte legame tra la figura del mortale e il monte della Caria. E infatti il monte Latmo è il teatro della tradizione che lega Endimione a Selene, il capostipite della tradizione discende da Saffo fr. 199V che ci informa anche dell'immortalità del pastore o bovaro a seconda delle versioni. 

La storia degli amori di Selene ed Endimione passa attraverso il mondo greco in maniera molto silenziosa. Le testimonianze di questa \textit{love story} sono infatti molto tarde. Sottolinea bene Gabelmann come il mito di Endimione, anzi il suo sonno, diventa <<sprichtw\"ortlich>>\footnote{LIMC III, 1, p. 727. Si vedano in merito le testimonianze di Plat. Phaid. 72C; Aristot. Eth. Nic. 10, 8.} fintanto che nella Suda si trova scritto <<\begin{otherlanguage}{greek} >Endum'iwnoc <'upnon kaje'udeic \end{otherlanguage}>>.

In particolare Luciano, Apollonio Rodio e Teocrito, quest'ulitmo probabilmente una delle fonti dell'approdo romano.\footnote{Lascio volutamente da parte il filone della tradizione unico e discendete da Licimno di Chio fr. 771 PMG, che fa di Endimione l'amato di Ipno e che collega il suo sonno direttamente al dio del Sonno. Mi sembra infatti questa una facile razionalizzazione, come ravvisibali anche nelle più tarde rappresentazioni funerarie dove Ipno appare spesso come intermediario tra Selene e il pastore (cfr. § 2.2.1).}

Luciano in particolare dedica all'amore di una eccitata Selene un dialogo con Afrodite.

Nel mondo romano Endimione, malgrado lo scetticismo di Cicerone, è a tutto tondo la figura del pastore di cui è innamorata la Luna. In Properzio, tanto quanto in Catullo (Callimaco, dunque Catullo), 




FONTI ENDIMIONE: RDK 5 (1967) 326-33 (LEOPOLD ETTLINGER): LIMC 3 (1986) 726-42 (HANNS GABELMANN), LIT. 728; DNP 3 (1997) 1027 (TANJA SCHEER); DNP SUPPL. 5 (2008) 253-7 (MARC F\"OCKING), LIT. 257

 \section{Iconografia antica greca e latina}
 D'altro canto sarebbe suggestiva l'ipotesi etimologica. Il nome di Endimione, dall'etimo ovviamente incerto, di Kerényi che lo fa discendere dal verbo ἐνδύω o ἐνδύνω, il cui significato letterale di ``stare dentro'' è propriamente usato per i vestiti (coprirsi, vestirsi etc.) e interpretato da Kerényi come <<jemanden, der sich innen befindet, umfa\ss t von der Geliebten, wie in einem gemeinsamen Kleid>>,\footnote{\cite{Kerenyi}, vol.1, p. 193.} Quest'etimo, che si accetti o meno l'interpretazione poetica di Kerényi, ben si accorda con le raffigurazioni tombali che riportano il mito di Endimione. Anche l'interpretazione di L\"ubker che ritiene il nome essere  la personificazione del sonno che coglie all'improvviso nel <<monte dell'oblio>>, il Latmo, interpretato con psilosi per Λάθμος, da λανθάνω.\footnote{Lessico ragionato dell'antichità class., s. v. Endymion, etimologia tuttavia poco attendibile. [controlla citazione].} riconosce al mito una funzione escatologica. 
 
 Paul Zanker in un libro che fa scuola riguardo alle rappresentazioni mitologiche sui sarcofagi dedica un sottocapitolo al mito di Selene ed Endimione.\footnote{\cite{Zanker}, pp. 316-24.}. Esistono solo tra l'età di Adriano (117-38) e quella gallienica (253-60 d. C.) circa 110 sarcofagi raffiguranti il mito di Endimione.\footnote{Si veda in merito anche \cite{Sichtermann}; LIMC III.1, pp. 726-42 a c. di Hanns Gabelmann, RDK 5, pp. 326-33.} L'iconografia endimionica e selenica è tuttavia ben più antica e risalente almeno al V secolo prima di Cristo, così come testimoniato da un vaso attico a figure rosse, il cratere <<Blacas>> del British Museum.\footnote{LIMC, II, 1, pp. 910-1, n. 22 = LIMC III, 1, p. 729, n. 11.} Il vaso raffigura, proveniente dalla Puglia, l'epifania del Sole a bordo di un carro trainato da 4 cavalli alati, diametralmente opposto all'immagine di Selene che si ritira sul suo cavallo. Da entrambe le parti Endimione si trova nell'atto dell'\textit{aposkopein}.\footnote{\cite{Jucker}, pp. 38-9. Allo stesso modo si presenta un'altra testimonianza vascolare, il cratere conservato all'Ermitage di San Pietroburgo, raffigurante Elio, la Notte, Selene a cavallo ed Endimione dietro col braccio alzato in posa di \textit{aposkopein}. Il secolo è sempre il V  } Di fianco ancora Eos la cui attenzione è attratta da un altro giovane Orione, Titono o Cefalo e la personificazione di alcune stelle. (Fig. 1) 
 
 Più interessante invece un specchietto della seconda metà del IV secolo a. C. proveniente da Demetria (Fig. 2)\footnote{LIMC II, 1, p. 914, n. 55.} e conservato nel Museo Nazionale di Atene. Lo specchio rappresenta un'epifania di Selene (si veda la \textit{velificatio} in cui la dea è intenta) di fronte a Endimione, ancora sveglio e con alle spalle una Chlamys; a destra un Eros con in mano una fiaccola e a sinistra un cane.
 

  \begin{figure}\begin{center}
 \includegraphics[scale=0.1]{Blaka.jpg} \label{Fig. 1} \caption{Cratere Blacas, Londra, British Museum}
\includegraphics[scale=0.1]{Kleppspiegel.jpg} \label{Fig. 2} \caption{Specchietto raffigurante Selene ed Endimione, Atene, Museo Nazionale}
\end{center} \end{figure}


Queste due testimonianze sono importanti per almeno tre motivazioni: \textit{in primis} che tra V e IV secolo il mito di Endimione è già legato strettamente al nome di Selene, e che questa tradizione ha raggiunto buona parte della penisola elladica (Endimione appunto indossa la Chlamys nello specchio). Nel caso dello specchio ancora più importanti risultano sia il legame tra epifania selenica, amore (il piccolo Eros dietro) ed Endimione-pastore (il cane), sia il fatto che quest'iconografia sia legata, oltre a una funzione prettamente funeraria, alla questione della bellezza di Endimione. 
In qualche maniera infatti Endimione impersona sia la bellezza perfetta, quella amata dal dio, sia allo stesso modo rappresenta la morte della bellezza stessa. 

\subsection{Le rappresentazioni funerarie}

Come già citato Zanker sottolinea la grande quantità di rappresentazioni funerarie afferenti alla sfera del mito di Endimione. Endimione infatti insieme alla sua amata è a tutti gli effetti un mito escatologico. Allo stesso modo Sichtermann dedica un intero volume ai sarcofagi tardi con le rappresentazioni del nostro mito.

Carl Robert ha riunito nella sua grande raccolta tutte gli esemplari conosciuti rappresenanti il mito di Endimione e ha interpretato la predilezione della scelta, riprendendo Otto Jahn\footnote{\cite{Jahn}, pp. 51ss.}, per l'ascendenza del passo di Cicerone delle Tuscolane, tuttavia rimane in questa intepretazione uno scetticismo di fondo del motivo della scelta di questo mito.
Franz Cumont, nel lavoro capitale sul simbolismo funerario dei romani, di vedervi il prototipo del matrimonio eterno di due sposi in un'altra esistenza: <<Endymion serait l'homme que dans les heures nocturnes revoit en r\^eve sa femme défunte, et qui, un jour, ira la retrouver à jamais dans un autre monde>>.\footnote{\cite{Cumont}, p. 247.} È infatti credenza di ascendenza pitagorica che le anime, liberate dal corpo, nel mondo dei sogni possano conversare con le anime dei morti. [Questa cosa irtorna in Petrarca] Una satira di Cwrron che si ispira probabilmente a questa credenza fa di Endimione un anima che percorre la città per farne un resoconto dettagliato. L'idea alla base è ovviamente collegata alla filosofia platonica per cui la vita terrestre altro non è se non un sogno in cui l'anima, bloccata dal corpo, si nutre di illusioni e chimere, finché la morte non la libera per farla accedere alle realtà divine.\footnote{Cfr. Plat. Rep. VII, 534C; Philon. De Iosepho 23, 130; Plot. Enn. III, 6.6.}

Tertulliano (De Anima 55) invece la Luna è la stazione intermediaria dove le anime riposano, come Endimione, prima di arrivare al loro destino finale. Tale concezione è resa più esplicita da Plutarco che nel \textit{De face orbe lunae} che lo stato di sonno di Endimione è uno stato ingannatore, in quanto prostrati da un sonno, cercano una nuova incarnazione.
La vita corporale dell'anima è un sogno e la morte un risveglio.
Nel mondo romano, dice Cumont,\footnote{\cite{Cumont}, p. 249.} gli amori di Selene ed Endimione acquistano una portata morale che nel mondo greco non hanno e simbolizzano l'attrazione che la Luna esercita sulle anime che, dal canto loro, aspirano a raggiungere l'astro divino.
 

Koortbojian tratta dei sarcofagi di Endimione e descrive un sarcofago romano in cui Rhea Silvia ed Endimione sono accomunati iconograficamente. La sorte comune dei due mortali amati dagli dei uniti alla rappresenazione funeraria spinge a itnerpretare questa comunanza come una necessità umana di testimoniare come <<in death one will enjoy-- quite literally-- the love of the gods>>.\footnote{\cite{Koortbojian}, p. 110.}

<<Identifications with the Endymion myth, whit its endless nights of passion, imply that to be chosen by the gods is to be granted the gift of a love equally everlasting, a love that survives death. The monument depicting a couple as goddess and youth elevates their love to the plane myth and realizes the dream of such an eternal union. Indee, this is an image of \textit{heroic} love, The sepulchral image declares thiers to be that ``great love that passes beyond the shores of Fate'' (Prop. I 19 12)>>

L'idea di Venere psicompompa raisale a Tibullo (I 3 57ss), un ruolo generalmente riconosciuto solo ad Hermes
“According to Plotinus, the pure soul represented by Aphrodite Ourania had its rightful place in the heavens above. The love she fostered taught man to aspire to such heights and to find fulfillment in his contemplation of the gods. Aphrodite Pandemos, by contrast, was a uni Pausanias attests a similar division in the mythology of Aphrodite, who was worshiped in Attica as Aphrodite Epistrophia ("she who turns men to love") and was venerated at Thebes as Apostrophia ("she who turns one away from love").17 These contrasting aspects of the Greek goddess were echoed by specifically Roman traditions, most strikingly in the celebrations marking the first of April in the Roman calendar that were dedicated to both the cult of Venus Verticordia and that of Fortuna Virilis.18 The former cult, instituted to revive the mores of Rome when they had fallen from their former chastity,19 venerated that aspect of Venus who "turns the hearts" of the mulieres honestiores toward virtue.20 The contrasting celebrations of the cult of Fortuna Virilis, in which the mulieres humiliorescommon prostitutesshared a common bath with the men, displayed themselves naked, and drank aphrodisiacs, were the expression of rather different values.

“This pair of opposing allegories provided the structure reflected in the contrasted Aphrodites of the sarcophagi.22 This tradition allowed the goddess to be easily accommodated to the Romans' distinction between the heavenly and earthly realms. The eroticism customarily associated with the Adonis myththe eroticism of Aphrodite Pandemosplayed little role on the sarcophagi. Yet by the goddess's “actions as Aphrodite Ourania, Adonisthe mortal herowas revived and elevated as her equal.	
Selene and Endymion posed a somewhat different problem. On the majority of reliefs Selene appeared as an acolyte of Aphrodite, thus conforming to the heightened eroticism of the mythological narrative. Surrounded by erotes, her breast bared, she was cast in the role of the seducer, and thus iconographically associated with Aphrodite Pandemos.	
Yet the myth's visual tradition was characterized by a seeming contradiction. Selene could also be regarded as a reflection of Aphrodite Ourania.23 This was no doubt exaggerated in later periods by the assimilation of the Greek moon goddess to the famously chaste figure of Roman Diana.24 As the essential motif of the sleeping Endymion had come to be associated with that of the sleeping Ariadne and Rhea Silvia, all three myths came to serve for the representation of a peaceful rest in death. Because their divine encounters, particularly those of Endymion and Ariadne, were also construed as celestial "marriages," these mythological narratives were regarded as the equivalent of apotheosis.25	
The celestial-terrestrial dichotomy lived on in the Graeco-Roman world's reception and transformation of the ancient Greek myths and is found[…]  “ment of other tales as well. The same Platonic myth underlies Achilles Tatius's discussion of the two kinds of love and distinguishes the special fate of Ganymede from that of Zeus's other conquests.26 Its role in other myths testifies to its importance and fundamental significance. For it was undoubtedly as a descendant of Aphrodite Pandemos that Venus rewarded the prayers of Pygmalion by transforming Galatea and watching over "the marriage she had made."27 This same conception of the goddess of lovegleefully rewarding the baser instinctsappears in Apuleius's account of the Judgment of Paris.28 And it is surely as Pandemos that Aphrodite was the lover of Adonis, an interpretation the eroticism of the poets' treatment of the myth frankly declares, and which is borne out, if only subtly, on the majority of sarcophagi. The contrasting conception of Aphrodite Ourania continued to play a role as well, and thus she appears as an aspect of Isis in Apuleius.29
28. Apuleius, Metamorphoses, X.31; cf. C. Schlam, "Platonica in the Metamorphoses of Apuleius," TAPA 101 (1970).	
		
29.Metamorphoses, XI.2.	
		”

Passi di: test. “test”. iBooks. 

Passi di: test. “test”. iBooks. 
		
23. See Boedeker, Aphrodite's Entry into Greek Epic, p. 14, on the association of Aphrodite Ourania with Selene.	
		
24. Perhaps most subtly, by Quintus Smyrnaeus, X.125ff., where Selene is athanatos and (perhaps) akeratos. Yet see the warning about conflating Selene and Diana issued by Sichtermann, in ASR XII.2, pp. 35f.	
		
25. Wrede, Consecratio, p. 152, and cf., in general, pp. 158175.	
		
26. Achilles Tatius, Leucippe and Clitophon, II.36; see the discussion in L. Barkan, Transuming Passion: Ganymede and the Erotics of Humanism (Stanford, 1991), pp. 35f.	
		
27. Ovid, Metamorphoses, X.295.”

Passi di: test. “test”. iBooks. 

Passi di: test. “test”. iBooks. 

Sarcofago capitolino (quello della madre) 
Selene in segno epifanico accorre dal carro verso Endimione, alle spalle Aura, personificazione della brezza. Gli Eroti illuminano la via con fiaccole e la accompagnano all'amato che, nel frattempo è addormentato nelle braccia di Hypnos [Morfeo], dio del Sonno. Ipno è rappresentato con la barba e l'età del suo, sempre compagno, Thanatos. \footnote{\cite{Koorbojian}, p. 66} Ipno è intento a tirare un manto per coprire le parti intime di Endimione, coprendo la giovane sessualità del mortale. [è interessante perché in tutti i racconti di Endimione, oltre che essere sempre coperto nei sarcofagi, è sempre rappresentaot come un amore casto. Mai giace con Luna, ma sempre vien e baciato. Non c'è nessun tipo di sessualità. D'altro canto essendo brutali nun se po' fa se quello dorme.]


 Endimione, un `superbello' amato da una dea, che, alla stregua di tanti altri superbelli (Adone, Titono...) ha un destino peculiare, una quasi happy ending story rispetto ai suoi colleghi mortali, e proprio questo finale a lieto fine ne ha permesso una diffusione e soprattutto una vasta ripresa nel '500. La storia di Endimione nasce infatti come mito minore; nella prima parte della tesi analizzeremo i filoni principali di diffusione del mito nel mondo antico.
Libro della Agapiou, allieva di Brunel, una ripresa metodologica o un grande flop?
A chi si avvicina? alla trattazione su Narciso?
Quali sono le trattazioni mongrafiche sulla mitologia e quale metodologia viene utilizzata?
Qual è il logos di questo mythos?
Ci interessa davvero l'archetipo per la nostra analisi? Mettersi nei panni dei rinascimentali è giusto? Perché abbiamo scelto solo la questione neoplatonica? Chi ne parla e perché Delminio lo tira fuori.
Chi ne parla, chi ne ha scritto.
\section{La tradizione latina}
\section*{La storia di un mito minore}
Dalle origini, no carrellatta ma effettivi punti importanti per la nostra discussione
\section*{Diana vs Endimione}
\section*{Prime ri-attestazioni rinascimentali}
\chapter{Endimione di Delminio e <<la morte di bacio>>}
\url{https://books.google.de/books?id=AT4kDwAAQBAJ&pg=PA49&lpg=PA49&dq=Felix+Endymion&source=bl&ots=_kkIezb7AG&sig=Th5OUJjNRo1citKyW976WNZ7ilQ&hl=it&sa=X&ved=0ahUKEwi08u7w_d3VAhVBC8AKHf7UAz8Q6AEIPDAH#v=onepage&q&f=false}

Boccaccio e co
\section*{Endimione filosofico, un aspetto peculiare del mito}
Camillo Delminio e la morte di bacio, fonti del mito
\chapter*{Conclusioni}
\addcontentsline{toc}{chapter}{Conclusioni}
\chapter*{Appendice I}
Cima da Conegliano (1459/60-1517/8) dipinse tra il 1505 e il 1510 (data imprecisa e ancora molto dibattuta) due tondi speculari di dubbia commissione e destinazione. I due piccoli dipinti, che si trovano nella galleria nazionale di Parma,\footnote{Sono infatti noti come <<tondi parmensi>>.} rappresentano un Endimione dormiente (Fig. 1) e un giudizio di Mida (Fig. 2), raffigurante Pan, Mida e Apollo nell'atto di terminare la gara di musica. I due tondi furono commissionati sicuramente insieme, come testimoniato dalla analoga grandezza e dalla evidente specularità. Probabilmente erano destinati a una camera da letto o a una camera di musica (dati i due soggetti). Essi pongono particolari problemi sia per l'interpretazione, sia per l'effettiva destinazione. 
Cima da Conegliano ha dipinto quattro quadri a soggetto mitologico: i due tondi parmensi, le nozze di Bacco e Arianna, e un Orfeo di cui ci rimane solo l'abbozzo e le parole di Rilke. Un soggetto dunque peculiare nella produzione di Cima, ma che in virtù di questo presente un enigma interpretativo.

Si sono pronunciati in merito alcuni studiosi importanti, proponendo qua e là identificazioni, commissionamenti, significati. In particolare ricorodo Humphrey, Luisa Viola e  Francesco Gandolfo (i più famosi) e ultima fra tutti Natalia Agapiou che ha inserito la trattazione all'interno del libro sulla fortuna rinascimentale del mito di Endimione.  Ha inoltre, quest'ultima, ripubblicato la sua teoria in un articolo successivo, in cui fa un'\textit{errata corrige} alla trattazione del libro, per un'interpretazione sbagliata di uno dei tre animali presenti nel quadro.
\pagebreak
\section*{Abbreviazioni e Acronimi}
\addcontentsline{toc}{chapter}{Acronimi e Abbreviazioni}
\begin{acronym}[tuamamma]
\acro{GDLI}{\textit{Grande Dizionario della Lingua Italiana}, a cura di S. Battaglia, Torino, UTET, 2001, 21 voll.}
\acro{DNP}{\textsc{Cancik}, Hubert e Helmuth \textsc{Schneider} et al.,\textit{ Der neue Pauly. Enzyklopädie der Antike. Das klassische Altertum und seine Rezeptionsgeschichte}, Stuttgart, J.B. Metzler, 2003, 19 voll. }
\acro{LIMC}{\textit{Lexicon Iconographicum Mythologiae Classicae}, Z\"urich-M\"unchen, Artemis Verlag, 1999, 18 voll. Disponbile anche online al sito http://www.limc-france.fr/.}
\acro{RDK}{\textit{Reallexikon zur deutschen Kunstgeschichte}, Stuttgart/Stuttgart-Waldsee, Metzler/Druckenmüller, 1927- 2014, 10 voll. Disponibile anche online al sito http://www.rdklabor.de/wiki/Hauptseite.}
\acro{Treccani-900}{Treccani, Enciclopedia del '900 online: http://www.treccani.it/\\enciclopedia/tag/enciclopedia-del-novecento/}
\end{acronym}

\pagebreak
\addcontentsline{toc}{chapter}{Bibliografia}
\section*{Bibliografia}
\rhead{\textit{BIBLIOGRAFIA}}
\lhead{}
\printbibliography[heading=Strum, notkeyword = Fon]
\printbibliography[heading=Fon, keyword = Fon]
\printbibliography[heading=FonB, keyword = FonB]

\end{document}
Alles was in der Kultur t\"atig oder gegenst\"andlich ist, alles was in ihr Gestalt annimt oder Form ergreift, muS, um heill zu werden, durch Deutung aus jeden Augenblick von neuem heilig werden; jede Kulturhandlung ist letzten Endes Kulthandlung, jeder Kulturgegenstand Kultgegenstand. Jolles p. 14

Über die Archetypen des kollektiven Unbewussten