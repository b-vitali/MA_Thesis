\documentclass[10pt,a4paper]{article}
\usepackage[utf8]{inputenc}
\usepackage[english]{babel}
\usepackage{amsmath}
\usepackage{amsfonts}
\usepackage{amssymb}
\usepackage{graphicx}
\usepackage{hyperref}
\title{In situ monitoring\\ of the stopped muon flux at Mu2e}
\author{Candidate: B. Vitali\\ Supervisors: Prof. S. Donati, Dr. P. Murat}
\date{}
\begin{document}
\maketitle
\noindent The Mu2e experiment at Fermilab is one of the future experiments dedicated to the search for Charge Lepton Flavor Violation.
The specific object of Mu2e search is  the neutrino-less coherent $\mu^-\rightarrow e^-$ conversion in the field of an aluminum nucleus and the signal is a monoenergetic electron of energy $ \approx 104.97$ MeV \cite{MTDR}.
%The parameter often used to indicate the strength of CLFV transition is the ratio $R_{\mu e}= (\mu^-N \rightarrow e^-N)/(\mu^-N \rightarrow \text{all } \mu \text{ captures})$ .
This process is forbidden in the Standard Model but allowed in many of its extensions: with minimal changes to include neutrino masses and oscillation, the Branching Ratio of this process, or the similarly interesting $\mu\rightarrow e \gamma$ decay, is expected to be of the order of $\mathcal{O}(10^{-54})$.
Values like these are below any currently achievable experimental sensitivity, and the observation of a signal would be unambiguous evidence of New Physics \cite{Signorelli}.
The upper limit on the muon conversion was set by SINDRUM II at $7\times10^{-13}$ (90\% C.L.) \cite{SINDRUMII} and the goal of the Mu2e collaboration is an improvement of 4 order of magnitudes.\\
The Mu2e experiment can be conceptually divided in three stages: interaction of the primary proton pulse with the tungsten target and production of $\pi$ and $\mu$; collection and transport of the produced particles down to the aluminum stopping target; interaction with the stopping target and measurement of the output particles.
The measurement of the stopped muon flux is of cardinal importance for the normalization of the Mu2e results.
The baseline design includes two detectors developed to measure this flux, counting $\gamma$ emitted by the muonic atoms: HPGe and LaBr$_3$(Ce) \cite{STM:2016}\cite{LaBr3:2020}.
The studies performed by the Mu2e Collaboration show that these two systems can be reliably used to determine the overall normalization.\\
The number of stopped muons is proportional to the number of protons on target, which itself depends on the extraction system for the proton beam. 
The Mu2e beam delivery system will use a resonant extraction to create the proton pulses and this method is characterized by intensity fluctuations on the time scale of milliseconds \cite{SpillSim}. 
These fluctuations, aside from the proportional fluctuation of the number of muons, have a non trivial impact on the overall performance of the apparatus. 
For example, higher intensity proton bunches translate into higher veto rates by the Cosmic Rays Veto system, 
which is triggered by secondary particles generated by the proton interaction. 
Another example is the effect of the instantaneous luminosity on the reconstruction efficiency: higher intensity translates into higher detector occupancy, which in turn reduces the reconstruction efficiency.\\
The two cited detectors encounter limitations when trying to monitor the flux at the millisecond timescales: HPGe is by construction a slow detector while the rate of emission of the $\gamma (1809\textrm{ keV})$ the LaBr$_3$(Ce) will measure is too low.
As of today, no system allows to monitor these fluctuations and the goal of this Thesis is to try filling this gap.
The method we developed to monitor the fluctuations of the stopped muon flux, and by extension the fluctuation in the proton pulses intensity, is based on counting the number of muons captured in the stopping target by counting the number of protons produced in the process of the nuclear muon capture.
This is one of the possible processes a stopped muon can undergo and it can lead to the ejection of charged particles.\\
The reconstruction algorithms were developed by the Mu2e collaboration with the explicit (and sometimes implicit) aim to reconstruct tracks of electrons in an energy range of a conversion electron. 
The protons we are interested in behave quite differently from these electrons and the cardinal task of this study has been to tailor the reconstruction routines to these particles. 
This has been done by analyzing simulated events consisting of a single protons to optimize the procedure and understand the features of these tracks. 
A major difference with respect to the electrons is the non-relativistic nature of $100\div300$ MeV$/c$ protons: accounting for the low velocity is both a challenge and a unique signature of the tracks.\\
The use of a pulsed beam and the need to allow produced $\pi$ to decay and $\mu$ to interact in the stopping target forces the definition of a Mu2e \textit{event}: an event is comprised by everything which happens in a $1.7\ \mu$s window between the incidence two consecutive proton pulses.
This means that the reconstruction will be performed in an environment crowded by many different particles.
The next step of this study has been to estimate the performance of the method in a fully fledged Mu2e simulated event and to study what kind of selections would improve it.\\
Our results are satisfactory since the number of reconstructed protons is significant (few per event), and show that a monitor on the timescale of milliseconds is possible, with a statistical uncertainty below $5\%$.

\addcontentsline{toc}{chapter}{Bibliography}
\bibliographystyle{../custombib}%unsrt siam custombib
\bibliography{../bvitali_biblio}
\end{document}